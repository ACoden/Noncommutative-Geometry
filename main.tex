\documentclass{article}
\usepackage{gensymb}
\usepackage[english]{babel}
\usepackage[T1]{fontenc}
\usepackage[utf8]{inputenc}
\usepackage[usenames]{color}

\usepackage{amsmath} 
\usepackage{amssymb} 
\usepackage{enumerate} 
\usepackage{enumitem} 
\usepackage{float}
\usepackage{amsthm} 
\usepackage{hyperref}
\usepackage{tikz}
\usepackage{xcolor}
\newcommand{\red}[1]{\textcolor{red}{#1}}
\usepackage{mathtools}
\theoremstyle{plain}
\newtheorem{theorem}{Theorem}[subsection]

\theoremstyle{definition}
\newtheorem{definition}[theorem]{Definition}

\newtheorem{lemma}[theorem]{Lemma}

\newtheorem{proposition}[theorem]{Proposition}
\newtheorem*{proposition*}{Proposition}
\newtheorem*{remark*}{Remark}

\theoremstyle{definition}
\newtheorem{remark}[theorem]{Remark}
\newtheorem*{exercise}{Exercise}
\newtheorem{fact}[theorem]{Fact}

\newtheorem{corollary}[theorem]{Corollary}

\theoremstyle{definition}
\newtheorem{example}[theorem]{Example}
\usepackage{mathrsfs}
\usepackage[autostyle]{csquotes}  


\usepackage{tikz-cd}
\usepackage{fancyhdr}
\usepackage{geometry}
\geometry{left=2cm,right=2cm,top=3cm,bottom=3.5cm,headheight=5mm,footskip=3cm,paper=a4paper}

\pagestyle{fancy} \fancyhead{}
\fancyfoot[C]{\textsl{Noncommutative Geometry}}
\renewcommand{\footrulewidth}{0.4pt}
\fancyhead[L]{\leftmark}
\fancyhead[R]{\bfseries \thepage}

\fancypagestyle{plain}{
\fancyhf{}
\fancyfoot[C]{\textsl{Noncommutative Geometry}}
\renewcommand{\headrulewidth}{0pt}
\renewcommand{\footrulewidth}{0.4pt}
}

\newcommand{\Le}{\mathcal{L}_\in}
\newcommand{\free}{\mathsf{free}}
\newcommand{\wh}[1]{\widehat{#1}}
\newcommand{\pow}[2]{ \scalebox{1.15}{\ensuremath{\mathscr{P}}}_{#2} \! \left ( #1  \right)}
\newcommand{\trcl}{\mathsf{trcl}}
\renewcommand{\k}{\kappa}
\newcommand{\ZF}{\mathsf{ZF}}
\newcommand{\ZFC}{\mathsf{ZFC}}
\newcommand{\Ord}{\mathsf{Ord}}
\newcommand{\Extension}{\mathsf{Extensionality}}
\newcommand{\Separation}{\mathsf{Separation}}
\newcommand{\Replacement}{\mathsf{Replacement}}
\newcommand{\Collection}{\mathsf{Collection}}
\newcommand{\Foundation}{\mathsf{Foundation}}
\newcommand{\Power}{\mathsf{Power}}
\newcommand{\Infinity}{\mathsf{Infinity}}
\newcommand{\Union}{\mathsf{Union}}
\newcommand{\Pair}{\mathsf{Pair}}
\newcommand{\T}{\mathsf{T}}
\newcommand{\p}{\vec{p}}
\renewcommand{\phi}{\varphi}
\newcommand{\G}{\Gamma}
\newcommand{\Id}{\mathrm{Id}}
\newcommand{\Hom}{\mathrm{Hom}}
\newcommand{\End}{\mathrm{End}}
\newcommand{\Sat}{\mathrm{Sat}}
\newcommand{\Fml}{\mathrm{Fml}}
\newcommand{\dom}{\mathrm{dom}}
\newcommand{\OD}{\mathsf{OD}}
\newcommand{\HOD}{\mathsf{HOD}}
\newcommand{\ST}{\mathsf{ST}}
\newcommand{\ran}{\mathrm{ran}}
\renewcommand{\sp}{\mathrm{sp}}
\newcommand{\Sep}{\mathsf{Sep}}
\newcommand{\Rep}{\mathsf{Rep}}
\newcommand{\Def}{\mathrm{Def}}
\newcommand{\Ind}{\mathrm{Ind}}
\newcommand{\rank}{\mathrm{rank}}
\newcommand{\C}{\mathbb{C}}
\newcommand{\R}{\mathbb{R}}
\newcommand{\Z}{\mathbb{Z}}
\newcommand{\Q}{\mathbb{Q}}
\newcommand{\N}{\mathbb{N}}
\newcommand{\U}{\mathcal{U}}
\newcommand{\Co}{\mathcal{C}}
\newcommand{\E}{\mathcal{E}}
\newcommand{\F}{\mathcal{F}}
\newcommand{\K}{\mathcal{K}}
\newcommand{\simh}{\stackrel{h}{\sim}}
\newcommand{\simu}{\stackrel{u}{\sim}}
\newcommand{\Simh}{\stackrel{h}{\approx}}
\newcommand{\Simu}{\stackrel{u}{\approx}}
\newcommand{\Proj}{\mathrm{Proj}}
\newcommand{\mat}[1]{\begin{pmatrix} #1 & 0 \\ 0 & 0\end{pmatrix}}
\newcommand{\matm}[1]{\begin{pmatrix} 0 & 0 \\ 0 & #1\end{pmatrix}}
\newcommand{\coker}{\mathrm{coker}}
\newcommand{\ind}{\mathrm{Index}}
\newcommand{\B}{\mathcal{B}}
\newcommand{\Ell}{\mathrm{Ell}}
\renewcommand{\End}{\mathrm{End}}
\newcommand{\supp}{\mathrm{supp}}
\newcommand{\sptr}{(\mathcal{A},H,D)}
\newcommand{\Cl}{\mathrm{C}\ell}
\newcommand{\CL}{\mathbb{C}\ell}
\newcommand{\spinc}{\mathrm{spin}^c}
\newcommand{\SO}{\mathrm{SO}}
\newcommand{\ev}{\mathrm{ev}}
\renewcommand{\L}{\mathcal{L}}
\newcommand{\Tr}{\mathrm{Tr}}

\begin{document}
\tableofcontents
\newpage
\section{$C^\ast$-Algebras}
\subsection{Banach Algebras and $C^\ast$-algebras}
\begin{definition}
 A \emph{Banach algebra} is an algebra $A$ with a submultiplicative norm $\|\cdot\|\colon A\to [0,\infty)$ such that $(A,\|\cdot\|)$ is a Banach space. Submultiplicative means that $\|ab\|\leq\|a\|\|b\|$ for all $a,b\in A$.
\end{definition}

\begin{exercise}
 Multiplication is continuous with respect to the topology induced by $\|\cdot\|$ on $A$ and $A\times A$ (the latter with the product topology).
\end{exercise}

\noindent Classical example are $A=\B(X)$, the algebra of bounded operators on a Banach space $X$, with the operator norm or closed subalgebras of $\B(X)$. We often assume that $A$ is unital, since we can always consider $A\oplus\C$ if it isn't.

\begin{definition}
 If $A$ is a Banach algebra and $a\in A$, we define the \emph{spectrum of $a$} as $$\sp(a)=\{\lambda\in A\mid \lambda-a\text{ is not invertible}\}.$$ 
\end{definition}
\begin{definition}
 If $A$ is a Banach algebra and $a\in A$, we define the \emph{spectral radius of $a$} as $$r(a)=\sup_{\lambda\in\sp(a)}|\lambda|.$$
\end{definition}

\begin{theorem}
 For every Banach algebra $A$ and every $a\in A$, $\sp(a)$ is a nonempty compact subset of $\C$ and $r(a)\leq\|a\|$. Furthermore $$r(a)=\lim_{n\to\infty}\|a^n\|^{1/n}.$$
\end{theorem}

\begin{definition}
 If $A$ is a Banach algebra, we define a \emph{character of $A$} to be a nonzero homomorphism $\gamma\colon A\to\C$.
\end{definition}
\begin{definition}
 If $A$ is a Banach algebra, we define the \emph{spectrum $\wh{A}$ of $A$} as the set of all characters of $A$.
\end{definition}

\begin{fact}
 If $\gamma\colon A\to\C$ is a character then $\ker(\gamma)$ is a "regular maximal" ideal, in particular $\ker(\gamma)$ is a closed ideal. This ensures that $\gamma$ is bounded, hence continuous, so $\wh{A}\subseteq A^\ast$, where $A^\ast=\B(A\to\C)$ is the dual of $A$.
 
 \noindent In particular if we consider a topology on $A^\ast$ we get an induced topology on $\wh{A}$. We will consider $A^\ast$ as a locally convex vector space with the weak-$\ast$ topology. Remember that for $\varphi_\lambda,\varphi\in A^\ast$ we have $$\begin{tikzcd}\varphi_\lambda\arrow["\text{w-}\ast",r] & \varphi\iff\varphi_\lambda(a)\arrow[r] &\varphi(a)\end{tikzcd},$$ for all $a\in A$.
\end{fact}

\begin{lemma}
 If $A$ is a Banach algebra then $A^\ast$ with the weak-$\ast$ topology is an Hausdorff space.
\end{lemma}
\begin{proof}
 Let $\varphi_1,\varphi_2\in A^\ast$ be two distinct functionals. Then there is an $a\in A$ with $\varphi_1(a)\neq\varphi_2(a)$. Let $\mathcal O_1$ and $\mathcal O_2$ be open disjoint nbhds of $\varphi_1(a)$ and $\varphi_2(a)$ in $\C$ and let $$\U_j=\mathrm{ev}_a^{-1}(\mathcal O_j)=\{\varphi\in A^\ast\mid\varphi(a)\in\mathcal O_j\}.$$
 In particular $\U_1$ and $\U_2$ are weak-$\ast$ open and separate $\varphi_1$ and $\varphi_2$, which concludes the proof.
\end{proof}

\noindent We can now consider $\wh{A}$ as a topological space with the subspace topology inherited from $A^\ast$.

\begin{proposition}
 If $A$ is a Banach algebra, then $\wh{A}$ is locally compact. If $A$ is also unital then $\wh{A}$ is compact.
\end{proposition}
\begin{proof}
 The closed unit ball $B^\ast\subseteq A^\ast$ is weak-$\ast$ compact by the Banach-Alaoglu theorem and $\wh{A}\cup\{0\}\subseteq B^\ast$.
 \red{THE FIRST LECTURE IS INCOMPLETE AND NEEDS TO BE WRITTEN FROM HERE}
\end{proof}
\section{Serre-Swan Duality}
\subsection{Vector Bundles}
In this section $X$ will denote a compact Hausdorff space
\begin{definition}
 A \emph{(complex) vector bundle} over $X$ is a topological space $E$ with a surjective continuous map $p\colon E\to X$ such that
 \begin{enumerate}
  \item $E_x=p^{-1}(x)$ is a vector space for all $x\in X$.
  \item $E$ is locally trivial, that is for all $x\in X$ there is an open set $\U$ containing $x$, $k\in\N$ and a homeomorphism $\phi\colon p^{-1}(\U)\to\U\times\C^k$ such that $\phi_x:E_x\to\C^k$ is linear and $p=\mathrm{pr}\circ\phi$ on $p^{-1}(\U)$, where $\mathrm{pr}:\U\times\C^k\to\U$ is the projection on the first factor.
 \end{enumerate}
\end{definition}

\begin{definition}
 A \emph{morphism of vector bundles} from $p\colon E\to X$ to $p'\colon E'\to X$ is a continuous map $\tau\colon E\to E'$ such that $p'\circ\tau=p$ and $\tau_x:E_x\to E_x'$ is linear for all $x\in X$.
\end{definition}

\begin{definition}\label{def: Gamma}
 The \emph{space of sections of $E$} is $\G(E)=\{\text{continuous maps $s\colon X\to E$ such that $p\circ s=\mathrm{Id}_X$}\}$.
 $\G(E)$ has the structure of a right $\Co(X)$-module with $$(sf)(x)=s(x)f(x),\quad\text{ for all }f\in\Co(X),s\in\G(E).$$
\end{definition}

\begin{lemma}\label{lemma: split}
 Any short exact sequence of vector bundles splits. In other words if $$\begin{tikzcd}0\arrow[r] & E'\arrow["\alpha",r] & E \arrow["\beta",r] & E''\arrow[r] & 0 \end{tikzcd}$$ is a short exact sequence then there exist a morphism $\sigma\colon E''\to E$ such that $\beta\circ\sigma=\mathrm{Id}_{E''}$. 
\end{lemma}

\begin{proof}[Proof (sketch):]\noindent
 \begin{enumerate}
  \item Any short exact sequence of vector spaces splits, this is a well known linear algebra fact.
  \item Using local triviality and continuity for each $x\in X$ there is an open neighbourhood $\U_x\subseteq X$ trivializing $E$ such that $$\begin{tikzcd}0\arrow[r] & E'_{|_{\U_x}}\arrow["\alpha",r] & E_{|_{\U_x}} \arrow["\beta",r] & E''_{|_{\U_x}}\arrow[r] & 0 \end{tikzcd}$$ is a short exact sequence.
  \item Pick a finite subcover and use a partition of unity argument to obtain a global splitting.
 \end{enumerate}
\end{proof}
 \begin{remark}
  Given $\beta\colon E\to E''$, $\ker(\beta)$ is a vector bundle iff $x\mapsto\rank(\beta_x)$ is locally constant. For example this holds whenever $\beta$ is injective or surjective. In general $\rank(\beta_x)$ is lower-semicontinuous, i.e. $$\forall x\in X\exists\,\U\text{ open nbhd of }x\text{ such that }\forall y\in\U\,(\rank(\beta_i)\geq\rank(\beta_x)).$$
 \end{remark}
 
\begin{proposition}\label{prop: direct sum}
 Given a vector bundle $E\to X$, there exist a vector bundle $F\to X$ such that $E\oplus F\simeq \C^n\times X$.
\end{proposition}
\begin{proof}[Proof (sketch):]
 Pick a finite subcover of trivializing nbhds $\{\U_j\}_{j=1}^n$ with a partition of unity $\{\chi_j\}_{j=1}^n$. For each $j$ we have $E_{|_{\U_j}}\simeq\U_j\times\C^k$ and there exist sections $s_{jl}\colon \U_j\to E$, $l=1,\ldots,k$ such that $s_{jl}(x)\in E_x$ is a basis for every $x$. Now define $\varphi_{jl}\colon X\to E$ as $$\varphi_{jl}=\begin{cases}
\chi_js_{jl} & \text{on } \U_j \\
0 &\text{otherwise}                                                                                                                                                                                                                                        \end{cases}$$

\noindent Let $m=nk$ and define $\beta\colon X\times\C^m\to E$ by $$\beta(x,v)=\sum_{j,l}v_{jl}\varphi_{jl}(x).$$
Then $\beta$ is surjective and we get an exact sequence 
$$\begin{tikzcd}0\arrow[r] & \ker\beta\arrow[r] & X\times\C^m \arrow[r] & E\arrow[r] & 0 \end{tikzcd}$$
which splits by Lemma \eqref{lemma: split}, hence $E\oplus\ker\beta\simeq\C^m\times X$.
\end{proof}

\subsection{The functor $\G$}

For each vector bundle $E\to X$ we obtain a $\Co(X)$-module $\G(E)$ as described in Definition \eqref{def: Gamma}. Moreover given a morphism $\tau\colon E\to E'$ of vector bundles over $X$ we obtain a $\Co(X)$-linear map $G(\tau)\colon\G(E)\to\G(E')$ given by $\G(\tau)(s)=\tau\circ s$.

\noindent Since $\G(\Id_E)=\Id_{E'}$ and $G(\tau\circ\sigma)=\G(\tau)\circ\G(\sigma)$ we have the following
\begin{lemma}
 $\G$ is a functor between the category of vector bundles over $X$ with morphisms of vector bundles to the category of $\Co(X)$-modules with $\Co(X)$-linear maps.
\end{lemma}

\begin{proposition}\noindent 
 \begin{enumerate}
  \item $\G(E)\oplus\G(E')\simeq\G(E\oplus E')$.
  \item $\G(E^\ast)\simeq\G(E)^\ast$, where $E^\ast\to X$ is the dual vector bundle and $\G(E)^\ast=\mathrm{Hom}_{\Co(X)}(\G(E),\Co(X))$ is the dual $\Co(X)$-module.
  \item $\G(E)\otimes_{\Co(X)}\G(E')\simeq\G(E\otimes E')$, where $E\otimes E'\to X$ has fibers $E_x\otimes E_x'$.
 \end{enumerate}
\end{proposition}

\begin{proof}[Proof (sketch of 3):]
 For $s\in\G(E),s'\in\G(E')$ define $s\odot s'\in\G(E\otimes E')$ by $$(s\odot s')(x)=s(x)\otimes s'(x).$$ We need to show that $s\otimes s'\mapsto s\odot s'$ is an isomorphism, call this map $\Theta\colon \G(E)\otimes_{\Co(C)}\G(E')\to\G(E\otimes E')$.
 
 \begin{itemize}
  \item If $E$ and $E'$ are trivial over $\U\subseteq X$ take local bases $s_j$ and $s_j'$ (meaning local sections which are pointwise basis of the fibers) and check that $s_j\odot s_j'$ is a local basis for $E\otimes E'$. In particular if $E$ and $E'$ are globally trivial, that is $\U=x$, this is an isomorphism.
  \item In general take $F$ and $F'$ such that $E\oplus F\simeq X\times\C^k$ and $E'\oplus F'\simeq X\times C^{k'}$ (those exist by Proposition \eqref{prop: direct sum}). We obtain the following commutative diagram:
  $$\begin{tikzcd}
    \G(E)\otimes_{\Co(X)}\G(E') \arrow["\Theta",r] \arrow["\G(i\otimes i')"',d] & \G(E\oplus E') \arrow["\G(i'')",d] \\
    \G(E\oplus F)\otimes_{\Co(X)}\G(E'\oplus F') \arrow["\widetilde{\Theta}",r] \arrow["\G(\sigma)\otimes\G(\sigma')"',d] & \G((E\oplus F)\otimes(E'\oplus F'))\arrow["\G(\sigma'')",d] \\
    \G(E)\otimes_{\Co(X)}\G(E') \arrow["\Theta",r]  & \G(E\oplus E')
    \end{tikzcd}
 $$
 and we get that $\Theta$ is surjective by the top half and injective by the lower half.
 \end{itemize}
\end{proof}

\begin{corollary}
 Each $\Co(X)$-linear map $\G(E)\to\G(E')$ is obtained by applying $\G$ to some $\tau\colon E\to E'$ (in other words $\G$ is a full functor).
\end{corollary}
\begin{proof}
 A morphism $\tau$ is a section of $\mathrm{Hom}(E,E')=E^\ast\otimes E'\to X$ (which is the vector bundle with fibers $E_x^\ast\otimes E_x'$), in other words $\tau\in\G(E^\ast\otimes E)$. But now we get that a $\Co(X)$-linear map $\G(E)\to\G(E')$ belongs to $$\Hom_{\Co(X)}(\G(E),\G(E'))\simeq\G(E)^\ast\otimes_{\Co(X)}\G(E')\simeq\G(E^\ast)\otimes_{\Co(X)}\G(E'),$$
 $\G(\tau)$ belongs to the leftmost term, while $\Theta^{-1}(\tau)$ to the rightmost one and that concludes the proof.
\end{proof}
\begin{remark}
 $\G$ is also faithful: $\G(\tau)=\G(\sigma)\implies\tau=\sigma$.
\end{remark}
\begin{lemma}\label{lemma: Gamma is exact}
 $\G$ is exact.
\end{lemma}
\begin{proof}
 If $\tau\colon E\to E'$ has constant rank then $\G(\ker(\tau))\simeq\ker(\G(\tau))$ and $\G(\ran(\tau))\simeq\ran(\G(\tau))$. Now if $$\begin{tikzcd}0\arrow[r] & E'\arrow["\alpha",r] & E \arrow["\beta",r] & E''\arrow[r] & 0 \end{tikzcd}$$ is exact then $\ran(\G(\alpha))\simeq\G(\ran(\alpha))$ and $\G(\ker(\beta))\simeq\ker(\G(\beta))$, so 
 $$\begin{tikzcd}0\arrow[r] & \G(E')\arrow["\G(\alpha)",r] & \G(E) \arrow["\G(\beta)",r] & \G(E'')\arrow[r] & 0 \end{tikzcd}$$ is also exact. Moreover if $\sigma\colon E''\to E$ splits $\beta$, then $\G(\sigma)$ splits $\G(\beta)$.
\end{proof}

\subsection{Modules}
In this section $A$ will denote an unital ring (for example an algebra).

\begin{definition}
 A \emph{morphism of right $A$-modules} $\varphi\colon\E\to\E'$ is a linear map such that $\varphi(sa)=\varphi(s)a$ for all $s\in\E$ and $a\in A$.
\end{definition}

\begin{definition}
 A right $A$-module $\E$ is called
 \begin{itemize}
  \item \emph{Free} if it has an $A$-basis, i.e. a set of generators $T\subseteq\E$ such that $t_1a_1+\ldots+t_na_n=0$ implies $a_i=0$ for all $i$ and all $a_l\in A,t_m\in T$.
  \item \emph{Finitely generated} if it has a finite generating set.
  \item \emph{Projective} if for every morphism $\varphi:\E\to\mathcal G$ and a surjective morphism $\eta\colon\mathcal F\to\mathcal G$ there is a morphism $\psi\colon\E\to\mathcal F$ such that $\varphi=\eta\circ\psi$, where $\mathcal G$ and $\mathcal F$ are arbitrary right $A$-modules.
 \end{itemize} 
\end{definition}

\begin{lemma}\noindent 
 \begin{enumerate} \label{lemma: free}
  \item Every free module is projective.
  \item If $\E=\bigoplus_{j\in J}\E_j$, then $\E$ is projective iff $\E_j$ is projective for every $j\in J$.
 \end{enumerate}
\end{lemma}

\begin{proposition} 
The following are equivalent:
 \begin{enumerate}
  \item $\E$ is projective.
  \item There is a module $\E'$ such that $\E\oplus\E'$ is free.
  \item There exist a free module $\F$ and $p=p^2\in\End_A(\F)$ such that $\E\simeq p(\F)$.
 \end{enumerate}
\end{proposition}

\begin{proof}\noindent 
 \begin{itemize}
 \item $(1)\implies(2):$ If $\E$ is projective any exact sequence $$\begin{tikzcd}0\arrow[r] & \mathcal G'\arrow[r] & \mathcal G \arrow[r] & \E\arrow[r] & 0 \end{tikzcd}$$ splits since $\Id:\E\to\E$ lifts to a splitting $E\to\mathcal G$. Pick a generating set $\{x_j\}_{j\in J}\subseteq\E$ and let $\F$ be the free module with generators $\{\delta_j\}_{j\in J}$. Define $\eta\colon\F\to\E$ by $\delta_j\to x_j$. Then $$\begin{tikzcd}0\arrow[r] & \mathcal \ker\eta\arrow[r] & \mathcal \F \arrow[r] & \E\arrow[r] & 0 \end{tikzcd}$$ is exact, hence it splits and therefore $\E\oplus\ker\eta\simeq\F$.
 \item $(2)\implies(1):$ If $\E\oplus\E'$ is free then it is projective and so is $\E$, by Lemma \eqref{lemma: free}.
 \item $(2)\implies(3):$ Given a split exact sequence $$\begin{tikzcd}0\arrow[r] & \mathcal \E\arrow[r] & \mathcal \F \arrow[r] & \E'\arrow[r] & 0 \end{tikzcd}$$ with $\F\simeq\E\oplus\E'$ define $p\in\End_A(\F)$ by $\F\to\E'\to\F$, where the second map is the splitting. Then $\E'\simeq p(\F)$ and $p=p^2$.
 \item $(3)\implies(2):$ If $\E=p(\F)$ with $p=p^2$ then $\F\simeq\E\otimes(1-p)\F$.
 \end{itemize}
\end{proof}

\begin{corollary}\label{corollary: fgp}
 The following are equivalent:
 \begin{enumerate}
  \item $\E$ is finitely generated and projective (also denoted by fgp).
  \item There exist $\E'$ such that $\E\oplus\E'\simeq A^{\oplus n}$ for some finite $n$.
  \item $\E\simeq pA^{\oplus n}$ for some $p=p^2\in M_n(A)$.
 \end{enumerate}
\end{corollary}

\subsection{Serre-Swan Duality}
\begin{proposition}
 Given a vector bundle $E\to X$, $\G(E)$ is fgp.
\end{proposition}
\begin{proof}
 By Proposition \eqref{prop: direct sum}here exist an $F\to X$ such that $$\begin{tikzcd} 0\arrow[r] & F\arrow[r] & X\times\C^n\arrow[r] & E\arrow[r] & 0\end{tikzcd}$$
 is split exact. By Lemma \eqref{lemma: Gamma is exact} 
 $$\begin{tikzcd} 0\arrow[r] & \G(F)\arrow[r] & \Co(X)^{\oplus n}\arrow[r] & \G(E)\arrow[r] & 0\end{tikzcd}$$ is also split exact and we can conclude applying $(2)\implies(1)$ from Corollary \eqref{corollary: fgp}
\end{proof}

\begin{theorem}[Serre-Swan]
 Every fgp $\Co(X)$-module is of the form $\G(E)$ for some vector bundle $E\to X$. Thus $\G$ is an equivalence of categories between vector bundles over $X$ with morphisms of vector bundles and fgp $\Co(X)$-modules with $\Co(X)$-linear maps.
\end{theorem}
\begin{proof}
 Let $\E$ be a fgp $\Co(X)$-module. Then there is $p=p^2\in M_m(\Co(X))$ such that $\E=p\Co(X)^{\oplus m}$ and $$\begin{tikzcd} 0\arrow[r] & \ker p\arrow[r] & \Co(X)^{\oplus m}\arrow[r] & \E\arrow[r] & 0\end{tikzcd}$$ is split exact, by Corollary \eqref{corollary: fgp}. Since $\G$ is full and faithful $p:\Co(X)^{\oplus m}\to\Co(X)^{\oplus m}$ induces a morphism of vector bundles $\tau\colon X\times \C^m\to X\times \C^m$.
 
 \noindent Now note that the functions $x\mapsto\rank(\tau_x)$ and $x\mapsto\rank(1-\tau_x)=m-\rank(\tau_x)$, where the last equality holds because $\tau_x$ is idempotent, are both lower semicontinuous. Hence $\tau_x$ is continuous and in particular it must be locally constant. So $\ran(\tau)$ is a subbundle of $X\times\C^m$ and $$\G(\ran(\tau))=\{\tau\circ s\mid s\in\G(X\times \C^m)\}\simeq\ran(p)\simeq\E.$$
 
 \noindent For the second part recall that $\G$ is an equivalence of categories iff it is fully faithful (already proved) and essentially surjective, which we've just shown.
 
 \noindent \red{TODO: I don't think this proof is very clear as written and details should be added.}
\end{proof}
\section{K-Theory}
K-theory is a cohomology theory that will give us topological invariants associated to a $C^\ast$-algebra. Since we will only obtain a semigroup rather than a cohomology group we first need to introduce an algebraic construction, known as the Grothendieck group which is, in some sense, "the most general" way to turn a semigroup into a group.

\subsection{The Grothendieck Group}
\begin{definition}
 Given a unital commutative semigroup $S$, the \emph{Grothendieck Group} of $S$ is the \emph{universal enveloping Abelian group} $G(S)$ defined as $$G(S)=S\times S/\sim=\{[(s,t)]\mid s,t\in S\},$$
 where $(s_1,t_1)\sim(s_2,t_2)\iff\exists r\in S(s_1+t_2+r=s_2+t_1+r)$.
\end{definition}
\begin{remark}
 Note that if $S$ has cancellation, meaning that $s+r=t+r\implies s=t$, for all $s,r,t\in S$, then we can forget about $r$ in the definition of $\sim$, but it is needed in general to ensure that $\sim$ is transitive (this similar to the definition of a localization for integral domain vs generic commutative rings with unity).
\end{remark}

\noindent The intuitive idea is to think of $[(s,t)]$ as $s-t$. This leads naturally to the following
\begin{definition}
 If $[(s_1,t_1)],[(s_2,t_2)]\in G(S)$, then \begin{align*}
                                             [(s_1,t_1)]+[(s_2,t_2)]&=[(s_1+s_2,t_1+t_2)] \\
                                             -[(s,t)]&=[(t,s)]
                                            \end{align*}

\end{definition}

\noindent There is a natural map $S\to G(S)$ given by $s\mapsto [(s,0)]$, which we will also denote simply by $[s]$ and this map is injective iff $S$ has cancellation. Note that $[(s,t)]=[s]-[t]$. The Grothendieck group $G(S)$ satisfies the following universal property: For any abelian group $H$ and any semigroup homomorphism $\varphi\colon S\to H$ there exist a map $\widetilde{\varphi}\colon G(S)\to H$ such that the diagram
$$\begin{tikzcd}
   S\arrow["\varphi",r] \arrow[d] & H \\
   G(S) \arrow["\widetilde{\varphi}"',ur,dashrightarrow]
  \end{tikzcd}
$$
commutes.

\noindent For a concrete example if $S=\N$ then $G(S)=\Z$.

\subsection{Topological K-Theory}
If $X$ is a compact Hausdorff space then the set of isomorphism calsses of $\C$-vector bundles over $X$, denoted with $V(X)$, is an abelian semigroup with direct sum as operation and with identity the class of the trivial $0$-dimensional bundle $X\to X$.

\begin{definition}
 The \emph{(even) K-theory of $X$} is defined as $K^0(X)=G(V(X))$.
\end{definition}
\begin{definition}
 Given a continuous map $f\colon X\to Y$ and a vector bundle $\pi\colon E\to Y$ we can defined the \emph{pullback bundle} $f^\ast E\to X$ with fibers $(f^\ast E)_x=E_{f(x)}$ as $$f^\ast E=\{(v,x)\in E\times X\mid \pi(v)=f(x)\}.$$
\noindent Since $f^\ast E\subseteq E\times X$ the map $f^\ast E\to X$ is simply the restriction of the projection $E\times X\to X$. Note also that given $s\in\G(E)$ we can get $s'\in\G(f^\ast E)$ by setting $s'(x)=((s\circ f)(x),x)$.
 
\noindent In particular a map $f\colon X\to Y$ induces a map $K^0(f)=f^\ast\colon K^0(Y)\to K^0(X)$ by $[E]\mapsto [f^\ast E]$.
\end{definition}
\begin{fact}
 $K^0$ is a contravariant functor from the category of compact Hausdorff spaces and continuous maps to the category of Abelian groups and group homomorphisms.
\end{fact}

\noindent For example if $X=\{\bullet\}$ is a space with a single point we get $V(X)=\N$ (there's one isomorphism class for every possible dimension of the vector bundle which is just a vector space) and so $K^0(X)=\Z$

\subsection{K-Theory for $C^\ast$-algebras}
By Serre-Swan the natural analogue in the case of a $C^\ast$-algebra $A$ is to consider the isomorphism classes of fgp $A$-modules over $A$, denoted with $V(A)$, which is a semigroup with addition induced by the direct sum of modules and the trivial module $0$ as identity element. Keeping the analogy with topological spaces we have
\begin{definition}
 The \emph{(even) K-theory of $A$} is defined as $K_0(A)=G(V(A))$. 
\end{definition}

\noindent Once again given any homomorphism $\varphi\colon A\to B$ for a $C^\ast$ algebra $B$ and a fgp $A$-module $\E$ we want to construct a pushforward $B$-module, to turn $K_0$ into a functor. We know that there is an idempotent $e\in M_n(A)$ such that $\E\simeq eA^{\oplus n}$ by Corollary \eqref{corollary: fgp}. Since $\varphi$ extends to $M_n(A)\to M_n(B)$ we obtain an idempotent $\varphi(e)\in M_n(B)$, so we can define $\varphi_\ast(\E)=\varphi(e)B^{\oplus n}$. Thus $\varphi\colon A\to B$ induces a map $K_0(\varphi)=\varphi_\ast\colon K_0(A)\to K_0(B)$ by $[\E]\mapsto[\varphi_\ast(\E)]$.

\begin{fact}
 $K_0$ is a functor from the category of unital $C^\ast$-algebras to the category of Abelian groups.
\end{fact}

\begin{lemma}
 By Serre-Swan we obtain $K_0(\Co(X))\simeq K^0(X)$.
\end{lemma}

\subsection{K-Theory through projections}
\begin{definition}
 Given a $C^\ast$-algebra $A$ a \emph{projection} in $A$ is a self-adjoint idempotent, i.e. a $p\in A$ with $p=p^2=p^\ast$.
\end{definition}

\begin{definition}
 Consider the following equivalence relations for projections $p,q\in A$:
 \begin{itemize}
  \item Homotopy: $p\simh q$ iff $\exists e=e^\ast=e^2\in\Co([0,1],A)$ such that $e(0)=p$ and $e(1)=q$.
  \item Unitary: $p\simu q$ iff $\exists u\in A$ with $uu^\ast=1=u^\ast u$ and $q=upu^\ast$.
  \item Murray-Von Neumann: $p\sim q$ iff $\exists v\in A$ such that $v^\ast v=p$ and $vv^\ast=q$.
 \end{itemize}
 Note that in the last definition we can assume wlog that $v=vv^\ast v=qv=vp$ (if not replace $v$ with $qv$).
\end{definition}

\begin{proposition}
 $p\simh q\implies p\simu q\implies p\sim q$.
\end{proposition}
\begin{proof}
 For the first implication we can assume wlog that $\|p-q\|<1$ (pick a continuous path joining $p$ and $q$ and split it into pieces of length less than $1$) and define $$r=\frac12((2q-1)(2p-1)+1).$$ Then $qr=qp=rp$ and $r^\ast rp=r^\ast qr=(qr)^\ast r=(rp)^\ast=pr^\ast r$ so $p$ commutes with $r^\ast r$ and hence also with $|r|=(r^\ast r)^{1/2}$. Furthermore $r$ is invertible:
 $$\|r-1\|=\|2qp-q-p\|=\|(q-p)(2p-1)\|\leq\|q-p\|<1$$
 where the inequality before the last one holds because $\|2p-1\|\leq 1$, since $p$ is a projection.
 
 \noindent Now take $u=r|r|^{-1}$. Then we have $upu^\ast=rp|r|^{-2}r^\ast=qrr^{-1}=q$, as desired.
 
 \noindent For the second implication suppose $q=upu^\ast$ and take $v=up$. It is easy to see that $v$ satisfies the necessary equalities.
\end{proof}

\noindent The reverse implications are not true, but we can reverse them if we allow for "more room".
\begin{proposition}\noindent 
 \begin{enumerate}
  \item $p\sim q\implies \begin{pmatrix} p & 0 \\ 0 & 0\end{pmatrix}\simu \begin{pmatrix} q & 0 \\ 0 & 0\end{pmatrix}$.
  \item $p\simu q\implies\begin{pmatrix} p & 0 \\ 0 & 0\end{pmatrix}\simh \begin{pmatrix} q & 0 \\ 0 & 0\end{pmatrix}$.
 \end{enumerate}
\end{proposition}
\begin{proof}\noindent 
 \begin{enumerate}
  \item Let $p=v^\ast v$, $q=vv^\ast$. Assume that $v$ is a partial isometry, i.e. $v=vv^\ast v=vp=qv$. Consider the unitary $$u=\begin{pmatrix} 1-q & v \\ v^\ast & 1-p\end{pmatrix}\begin{pmatrix} 1-p & p \\ p & 1-p\end{pmatrix}$$

 Then $$u\begin{pmatrix} p & 0 \\ 0 & 0\end{pmatrix}u^\ast=\begin{pmatrix} q & 0 \\ 0 & 0\end{pmatrix}.$$
 \item If $q=upu^\ast$, consider $$v_t=\begin{pmatrix} \cos(\frac{\pi t}2) & \sin(\frac{\pi t}2) \\ \sin(\frac{\pi t}2) & \cos(\frac{\pi t}2)\end{pmatrix}\quad \text{ and }w_t=v_t\begin{pmatrix} u & 0 \\ 0 & 1\end{pmatrix}v_{-t}\begin{pmatrix} 1 & 0 \\ 0 & u^\ast\end{pmatrix}.$$
 Let $p_t=w_t\begin{pmatrix} p & 0 \\ 0 & 0\end{pmatrix}w_t^\ast$. Then $p_0=\begin{pmatrix} q & 0 \\ 0 & 0\end{pmatrix}$ and $p_1=\begin{pmatrix} p & 0 \\ 0 & 0\end{pmatrix}.$
  \end{enumerate}
\end{proof}

\noindent Thus to make those equivalence relations the same we need to "stabilize". To make this idea precise consider the $C^\ast$ algebras $M_n(A)$ with the isometric inclusions $M_n(A)\hookrightarrow M_{n+1}(A)$, $a\mapsto\begin{pmatrix} a & 0\\0 & 0\end{pmatrix}$ and consider $$M_\infty(A)=\bigcup_{n\in\N}M_n(A).$$ Note that $M_\infty(A)$ is not a $C^\ast$-algebra since it isn't complete. 

\begin{definition}
 For $p\in M_m(A)$ and $q\in M_n(A)$ two projections we say that $p$ and $q$ are stably equivalent, denoted by $p\approx q$, iff $p\sim q$ in $M_N(A)$ for some $N$ large enough. Similarly for $\Simh$ and $\Simu$.
\end{definition}

\begin{corollary}
 The relations $\approx$,$\Simu$ and $\Simh$ coincide on the set $\Proj(M_\infty(A))$ of projections in $M_\infty(A)$.
\end{corollary}

\begin{definition}
 Let $K_0^+(A)$ denote $\Proj(M_\infty(A))/\approx$. Note that if we wanted to have a $C^\ast$-algebra we could equivalently take $\overline{M_\infty(A)}=A\otimes K$. 
\end{definition}

\noindent We can make $K_0^+(A)$ into an abelian semigroup by setting $[p]+[q]=\left[\begin{pmatrix}p & 0 \\ 0 & q\end{pmatrix}\right]$ and identity element $[0]$. This naturally brings us to

\begin{definition}
 The \emph{(even) K-theory of $A$} is $K_0(A)=G(K_0^+(A))$.
\end{definition}

We now look at some examples.
\begin{enumerate}
 \item $p,q\in M_\infty(\C)$ are stably equivalent iff $\rank(p)=\rank(q)$, hence $K_0^+(\C)=\N$ and $K_0(\C)=\Z$.
 \item Since $M_m(M_n(\C))=M_{mn}(\C)$ we have $M_\infty(M_n(\C))=M_\infty(\C)$ \red{why? aren't we missing some small ones?}, hence $K_0(M_n(\C))=\Z$ (this makes sense since $M_n(\C)\simeq \C^{n^2}$ is contractible).
 \item Let $K=\K(H)$ be the space of compact operators on an Hilbert space $H$. For any projection $p\in K$ we hanve $\rank(p)<\infty$ and again $K_0(K)=\Z$.
 \item If instead we look at $\B(H)$, the space of bounded operators on an Hilbert space $H$ and take a projection $p\in\B(H)$, we may have $\rank(p)=\infty$, hence $K_0^+(\B(H))=\N\cup\{\infty\}$ and, since $m+\infty=n+\infty$ for all $n,m\in\N$, we have $K_0(\B(H))=0$.
\end{enumerate}

\begin{proposition}\label{prop: homotopy}
 The set of idempotents in $A$ is homotopy equivalent to the set of projections. \red{I don't think homotopy equivalent is the correct term here, or at least I don't fully understand what it's meant}
\end{proposition}
\begin{proof}
 Given $e=e^2$ an idempotent in $A$ define
 \begin{itemize}
  \item An invertible positive \red{positive?} element $z=1+(e-e^\ast)(e^\ast-e)$ such that $ez=ee^\ast e=ze$ and $e^\ast z=e^\ast ee^\ast=ze^\ast$.
  \item A projection $p=ee^\ast z^{-1}$ such that $ep=p$ and $pe=e$.
  \item A continuous path of idempotents, for $t\in[0,1]$ let $$e_t=(1-tp-te)e(1+te-tp)$$ and note that $e_0=e$ and $e_1=p$.
 \end{itemize}
\end{proof}

\begin{corollary}
 For every fgp module $\E$ over $A$ there is $p=p^\ast=p^2\in M_n(A)$ such that $\E=pA^{\oplus n}$.
\end{corollary}
\begin{proof}
 We have $\E=eA^{\oplus n}$ for some $e=e^2$, by Corollary \eqref{corollary: fgp}. By Proposition \eqref{prop: homotopy} we can find $p=p^\ast=p^2$ with $pe=e$ and $ep=p$. Then we have 
 \begin{align*} 
  \E=eA{^\oplus n}=peA^{\oplus n}&\subseteq pA^{\oplus n} \\
  pA^{\oplus n}=epA^{\oplus n}\subseteq e&A^{\oplus n}=\E
 \end{align*}
\end{proof}

\subsection{Equivalence of the two definitions of K-theory}
\noindent We now have two definitions of $K_0(A)$ for a $C^\ast$-algebra $A$, through isomorphism classes of fgp modules over $A$ and through projections up to stable equivalence. We also have maps $$\begin{tikzcd}\{\text{fpg $A$-modules}\}\arrow["\E=eA^{\oplus n}",r]&\{\text{idempotents in $M_n(A)$}\}\arrow["\text{homotopy}",r]&\{\text{projections in $M_\infty(A)$}\} \end{tikzcd}$$
and it's natural to ask whether starting with two isomorphic modules gives us two stably equivalent projections and viceversa.
\begin{theorem}
 The map $K_0^+(A)\to V(A)$, $[p]\mapsto[pA^{\oplus n}]$ is an isomorphism. In particular $K_0(A)=G(V(A))\simeq G(K_0^+(A))$.
\end{theorem}
\begin{proof}
 One direction is easy: suppose $p,q\in M_\infty(A)$ are stably equivalent projections, in particular $q=upu^\ast$ for some unitary $u$. Then $u\colon pA^{\oplus m}\to qA^{\oplus m}$ is the desired isomorphism.
 
 \noindent The converse direction requires more work. Suppose that $\E$ and $\F$ are isomorphic fgp $A$-modules, so there exist projection $p,q\in M_N(A)$ with $pA^{\oplus N}\simeq\E\simeq\F\simeq qA^{\oplus N}$. We want to show that this implies $p\approx q$. Let $\varphi$ denote the map $pA^{\oplus N}\to qA^{\oplus N}$ obtained composing the isomorphisms above. We obtain $g,h\in M_N(A)$ given, for $v\in A^{\oplus N}$, by $gv=\varphi(pv)$ and $hv=\varphi^{-1}(qv)$. Then $gh=p$, $hg=p$ and $g=gp=qg$, $h=hq=ph$. Let $$z=\begin{pmatrix}g & 1-q \\ 1-p & h\end{pmatrix}\in M_{2N}(A).$$
 Then $z$ is invertible and we have $$z^{-1}=\begin{pmatrix} h & 1-p \\ 1-q & g\end{pmatrix}\text{  and  } z\mat{p}z^{-1}=\mat{q}.$$
 We now want to replace $z$ with a unitary matrix. Let $u=z|z|^{-1}$, where $|z|=(z^\ast z)^{1/2}$. Then $|z|p|z|^{-1}=|z|z^{-1}qz|z|^{-1}=u^\ast q u=(u^\ast qu)^\ast=|z|^{-1}p|z|$, so $p$ commutes with $|z|^{-1}$ and $|z|$.
 
 \noindent But now $q=u|z|p|z|^{-1}u^\ast=upu^\ast$.
\end{proof}

\subsection{Standard picture of K-theory}
Note that given projections $p\in M_n(A),q\in M_m(A)$, there is $p'\in M_{n+m}(A)$ such that $[p]-[q]=[p']-[\Id_n]\in K_0(A)$. Indeed $$[p]-[q]=\left[\mat{p}\right]-\left[\matm{q}\right]=\left[\begin{pmatrix}p & 0 \\ 0 & \Id_n-q\end{pmatrix}\right]-\left[\matm{\Id_n}\right].$$

\begin{theorem}
 The functor $K_0$ from unital $C^\ast$-algebras to Abelian groups has the following properties:
 \begin{itemize}
  \item Homotopy invariance: if $A$ and $B$ are homotopy equivalent $C^\ast$-algebras. then $K_0(A)=K_0(B)$.
  \item Stability: $K_0(A\otimes K)\simeq K_0(A)$ where $K=\K(H)$ is an algebra of compact operators on a separable Hilbert space.
  \item Continuity: if $A_1\subseteq A_2\subseteq A_3\subseteq\cdots$ then $\varinjlim K_0(A_n)\simeq K_0(\varinjlim A_n)$.
  \item Half-Exactness: If $\begin{tikzcd}0\arrow[r]& J\arrow["i",r] & A \arrow["\pi",r] & B\arrow[r] & 0\end{tikzcd}$ is exact then $\begin{tikzcd}K_0(J)\arrow["K_0(i)",r] & K_0(A)\arrow["K_0(\pi)",r] & K_0(B)\end{tikzcd}$ is exact at $K_0(A)$.
 \end{itemize}
\end{theorem}

\noindent Note that in general $K_0(i)$ is not injective: pick $J=\K(H)$, $A=\B(H)$, then we saw that $K_0(J)=\Z$ and $K_0(A)=\{0\}$. In general $K_0(\pi)$ is not surjective because projections in $B$ cannot always be lifted to projections in $A$.

\begin{remark}
 In the formulation of half-exactness we looked at a short exact sequence of $C^\ast$-algebras, those cannot all be unital. For a unital $C^\ast$-algebra $A$ we have $K_0(A)\simeq\ker(K_0(A\oplus\C)\to K_0(\C))$ so in the nonunital case we can use the right hand side as definition of the (reduced) K-theory of $A$.
\end{remark}
\subsection{Higher order K-theory groups}
Higher order K-theory groups can be defined via suspensions. Let $S(X)=X\times\Bbb R$ for a topological space $X$ and $S(A)=\Co_0(\Bbb R,A)$ for a $C^\ast$-algebra $A$. Then $K_1(A)=K_0(S(A))$ and $K_n(A)=K_0(S^n(A))$. It can also be described explicitely through projections as equivalence classes of unitaries in $M_N(A)$. \red{I'm not sure about the projections thing, but this should just be a side remark and we won't work with higher K-theory.}

\begin{theorem}[Bott periodicity]
 For all $n\in\N$, $K_n(A)\simeq K_{n+2}(A)$.
\end{theorem}

\section{Operators}
\subsection{Unbounded Operators}
\begin{definition}
 A \emph{densely defined operator} $D$ on a separable Hilbert space $H$ is a linear map $D\to H$, where $\dom D$ is dense in $H$.
\end{definition}

\begin{definition}
 The \emph{graph} of a densely defined operator $D$ is $$g(D)=\{(\psi,D\psi)\in H\oplus H\mid\psi\in\dom D\}$$
\end{definition}

\begin{definition}
 The \emph{adjoint} of $D$ is the operator $D^\ast$ with $$\dom D^\ast=\{\xi\in H\mid \exists\eta\in H\text{ such that }\forall\psi\in\dom D, \langle\xi, D\psi\rangle=\langle\eta,\psi\rangle,$$
 then $D^\ast\xi=\eta$.
\end{definition}

\begin{definition}
 A densely defined operator $D$ is called
 \begin{itemize}
  \item \emph{Closed} iff $g(D)$ is closed in $H\oplus H$. That is, if $\psi_n\in\dom D$ and $(\psi_n,D\psi_n)\to(\psi,\eta)$ in $H\oplus H$, then $\psi\in\dom D$ and $\eta=D\psi$.
  \item \emph{Closable} iff there exist a (necessarily unique) operator $\overline{D}$, called the \emph{closure of $D$}, such that $\overline{g(D)}=g(\overline{D})$.
  \item \emph{Symmetric} iff $\langle\varphi, D\psi\rangle=\langle D\varphi,\psi\rangle$ for all $\varphi,\psi\in\dom D$ (intuitively $D\subseteq D^\ast$).
  \item \emph{Self-Adjoint} iff $D$ is symmetric and $\dom D=\dom D^\ast$ (intuitively $D=D^\ast$).
  \item \emph{Essentially self-adjoint} iff $D$ is symmetric and $\dom D^\ast=\dom\overline{D}$ (intuitively $D^\ast=\overline{D}$).
 \end{itemize}
\end{definition}
\begin{fact}\noindent 
 \begin{enumerate}
  \item $D$ is closable iff $D^\ast$ is densely defined. Then $D^\ast$ is closed and $(D^\ast)^\ast=D$.
  \item Any symmetric operator is closable.
 \end{enumerate}
\end{fact}
\begin{theorem}
 A densely defined operator $D$ is self-adjoint iff $D\pm i\colon\dom D\to H$ are bijective.
\end{theorem}

\noindent We can equip $\dom D$ with the graph inner product $\langle\varphi,\psi\rangle_D=\langle\varphi,\psi\rangle+\langle D\varphi,D\psi\rangle$. Then $D$ is closed iff $\dom D$ is complete wrt to $\langle\cdot,\cdot\rangle_D$. Moreover $D\colon\dom D\to H$ is bounded wrt $\langle\cdot,\cdot\rangle_D$. \red{I'm not sure what the point of this paragraph is, should it be a remark or...?}

\subsection{Continuous Functional Calculus}
\begin{lemma}
 If $P$ is a positive operator, then $I+P$ is a bounded operator.
\end{lemma}
\begin{proof}
 We have $I+P-\lambda I=P+(1+\lambda) I$ so $\sp(I+P)$ is $\sp(P)$ shifted to the right by $1$, since $\sp(P)\subseteq[0,\infty)$ (because $P$ is positive), we have $0\not\in\sp(I+P)$, hence $I+P$ is invertible.
\end{proof}



\noindent Now if $D$ is a densely defined operator we have that $(1+D^2)^{-1}$ is a well defined and positive operator in $\B(H)$. Then we get $(1+D^2)^{-1/2}\in\B(H)$ such that $\ran(1+D^2)^{-1/2}=\dom D$ and hence we also get $F_D=D(1+D^2)^{-1/2}\in\B(H)$, a so called \emph{bounded transform} of $D$. We also get a commutative $C^\ast$-algebra $$C^\ast(1,F_D)\simeq\Co(\sp(F_D)),$$ with $\sp(F_D)\subseteq [-1,1]$ (This is the closure of all polynomials in $F_D$ in $\B(H)$, see also the remark after Theorem \eqref{thm: Gelfand-Naimark} for the isomorphism). 

\noindent Consider now an $f\in\Co_b(\Bbb R)$ such that $\lim_{x\to\pm\infty}f(x)$ exists and let $g(x)=f(x(1-x^2)^{-1/2})$, so in particular $g\in\Co([-1,1])$ and we can define $f(D)=g(F_D)\in C^\ast(1,F_D)\subseteq\B(H)$. This can be extended to an arbitrary $f\in\Co(\R)$. 

\subsection{Fredholm Operators}
\begin{definition}
 An operator $T\in\B(H)$ is called \emph{Fredholm} if $\ker T$ and $\coker T=H/\ran T$ are both finite dimensional. We denote the set of all Fredholm operators on $H$ by $\F(H)$.
\end{definition}

\begin{definition}
 For $T\in\F(H)$ we define the \emph{index} of $T$ as $\ind T=\dim\ker T-\dim\coker T$.
\end{definition}
\begin{fact}
 The following facts are true: 
 \begin{enumerate}
  \item For $T\in\B(H)$, $\ker T^\ast=(\ran T)^\perp$.
  \item If $T\in\F(H)$, $\ran T$ is closed and therefore $\ran T=(\ran T)^{\perp\perp}$.
  \item $T$ is Fredholm iff $\ran T$ is closed and $\ker T$, $\ker T^\ast$ are finite dimensional.
  \item If $T\in\F(H)$, then $T^\ast\in\F(H)$ and $\ind T^\ast=-\ind T$.
  \item For $T,S\in\F(H)$, $\ind (T\oplus S)=\ind T+\ind S$ and $\ind(TS)=\ind T+\ind S$.
  \item For $K\in \K(H)$, $1+K\in\F(H)$ and $\ind(1+K)=0$.
 \end{enumerate}
\end{fact}

\noindent Note that this all makes sense for operators $H_1\to H_2$, so in particular it makes sense for closed operators $D\colon\dom D\to H$.

\begin{definition}
 The \emph{Calkin} algebra is $\Co(H)=\B(H)/\K(H)$, with quotient map $\pi\colon\B(H)\to\Co(H)$.
\end{definition}

\begin{theorem}[Atkinson]
 A bounded operator $T\in\B(H)$ is Fredholm iff $\pi(T)\in\Co(H)$ is invertible. In other words $T$ is Fredholm iff there exist a "parametrix" $S\in\B(H)$ such that $\Id-ST,\Id-TS\in\K(H)$. \red{is it clear that those two formulations of Atkinson's theorem are equivalent?}
\end{theorem}

\begin{corollary}\noindent
 \begin{enumerate}
  \item $\F(H)$ is open in $\B(H)$: given $T\in\F(H)$ there is $\varepsilon>0$ such that for all $S\in\B(H)$ with $\|S\|<\varepsilon$, $T+S\in\F(H)$.
  \item if $T\in\F(H)$ and $K\in\K(H)$, then $T+K\in\F(H)$ and $\ind(T+K)=\ind(T)$.
 \end{enumerate}
\end{corollary}

\begin{theorem}\label{thm: index_is_homotopy_invariant}
 The map $\ind\colon\F(H)\to\Z$ is locally constant. In particular if $[0,1]\to\F(H)$, $t\mapsto T_t$ is continuous, then $\ind \,T_0=\ind\, T_1$. Moreover it induces an isomorphism $[\F(H)]\to\Z$, where $[\F(H)]$ is the set of path components of $\F(H)$.
\end{theorem}

\begin{theorem}[Atiyah-Janich] 
 For $X$ compact Hausdorff we have an isomorphism $[X,\F(H)]\to K_0(X)$ given by "family index", where $[X,\F(H)]$ are homotopy classes of continuous maps $X\to\F(H)$
\end{theorem}


\section{K-homology}
K-theory is a (generalized) cohomology theory, hence there is a dual (generalized) homology theory developed by Whitehead and others in the 60s and 70s. One possible way to define K-homology is as follows:

\noindent Given a topological space $X$ we construct a "dual space" $DX$ and define the K-homology of $X$ as the K-theory of its dual space:
$$K_0(X)=K^0(DX).$$
Atiyah (1969) realized that K-homology can be represented by elliptic operators. $F\in\Ell(X)$ induces an "index pairing" 
$$\Ind_F\colon K^0(X)\to K^0(\bullet)\simeq\Z.$$ 
Hence any $\alpha\in K_0(X)$ also gives a map $K^0\to\Z$ and we have a homomorphism $$K_0(X)\to\Hom(K^0(X),\Z),$$ which is generally not an isomorphism. But, for each space $Y$, $F$ also induces a map $$Y-\Ind_F\colon K^0(X\times Y)\to K^0(Y).$$
In general there is a pairing called "slant product" $$K_0(X)\times K^0(X\times Y)\to K^0(Y).$$
In the specific case of $Y=DX$ there is a $\mu\in K^0(X\times DX)$ such that $$-\otimes\mu\colon K_0(X)\to K^0(DX)$$ is an 
isomorphism. Then we obtain $$DX-\Ind_F(\mu)\in K^0(DX)\simeq K_0(X)$$ and we call this element $[F]$. Thus we have a map 
$\Ell(X)\to K_0(X)$ and Atiyah proved that this map is surjective (it is not injective). Kasparov (1975) provided the appropriate notion of equivalence on $\Ell(X)$ and proved $\Ell(X)/\sim\simeq K_0(X)$. Later (1980) he introduced "bivariant K-theory groups" $KK(X,Y)$ such that \begin{align*} KK(X,\bullet)&\simeq K_0(X)\\KK(\bullet,Y)&\simeq K^0(Y). \end{align*}
After this brief historical overview we start formalizing the notions we talked about so far.

\subsection{Fredholm Modules}
\begin{definition}
 A \emph{Fredholm Module} over a unital $C^\ast$-algebra $A$ is a pair $(H,F)$ where $H$ is an Hilbert space equipped with a representation $\pi:A\to\B(H)$ and $F\in\B(H)$ is a self-adjoint bounded operator such that $1-F^2$ is compact and $[F,a]$ is compact for every $a\in A$, where $[F,a]=Fa-aF$ and we're identifying $A$ and $\pi(A)$. 
\end{definition}

\begin{remark}\noindent
 \begin{enumerate}
  \item If $F$ is a zeroth order pseudodifferential operator on a smooth compact manifold $M$, then $[F,a]$ is compact for $a\in\Co(M)$. \red{what is a zeroth order pseudodifferential operator?}
  \item Since $1-F^2$ is compact, $F$ is Fredholm, with parametrix $F$ itself. However $\ind(F)=0$, since $F=F^\ast$, which is a "boring" case, we will have to look at a different case in which we get a nonzero index. 
 \end{enumerate}
\end{remark}

\begin{definition}
A Fredholm module $(H,F)$ over a $C^\ast$-algebra $A$ is called \emph{even} if there exist a $\gamma\in\B(H)$, called the \emph{grading operator}, such that $\gamma^\ast=\gamma$, $\gamma^2=1$, $F\gamma=-\gamma F$ and $\gamma a=a\gamma$ for all $a\in A$.  
\end{definition}

\noindent The idea behind this definition is that if $(F,H)$ is an even Fredholm module over the $C^\ast$-algebra $A$ then we have a decomposition $H=H^+\oplus H^-$ where $H^{\pm}$ are the $\pm 1$-eigenspaces of $\gamma$ and we can write 
$$\gamma=\begin{pmatrix}1 & 0 \\ 0 & -1\end{pmatrix},\quad\quad F=\begin{pmatrix}0 & F^- \\ F^+ & 0\end{pmatrix}.$$
Now we can define the even (or odd) K-homology of $A$ as the isomorphism classes of even (or odd) Fredholm modules over $A$, so to each $C^\ast$-algebra $A$ we associate an Abelian group $K^0(A)$, which is actually a contravariant functor. Given $\varphi\colon A\to B$ we obtain $K^0\varphi=\varphi^\ast\colon K^0(B)\to K^0(A)$ by $(H,\pi,F)\mapsto (H,\pi\circ\varphi,F)$.

\subsection{Index Pairing}
Consider an even Fredholm module $(H,F)$ over a $C^\ast$-algebra $A$ and an element $[p]\in K_0(A)$, where $p\in M_k(A)$ is a projection.
\begin{definition}\label{defn: H_k}
 We define $H_k=H\otimes\C^k$ and $H_k^\pm=H^\pm\otimes\C^k$. Note that $p$ acts on $H_k$ by extending $\pi$. We also define $F_k=F\otimes\Id_k$ and $F_k^\pm=F^\pm\otimes\Id_k$.
\end{definition}

\begin{definition}
 With the notation introduced in Definition \eqref{defn: H_k} we define $$\langle [p],[F]\rangle=\ind(pF_k^+p\colon pH_k^+\to pH_k^-)$$. Furthermore we define $\Ind_F\colon K_0(A)\to\Z$ by $[p]\mapsto\langle[p],[F]\rangle$.
\end{definition}
\begin{proposition}
 $\Ind_F$ is well defined.
\end{proposition}
\begin{proof}
 First we check that $pF_k^+p$ is a Fredholm operator. We have 
 \begin{align*}
  (pF_kp)^2&=pF_kpf_kp\\&=p[F_k,p]F_kp\\&=p[F_k,p]F_kp+p(F^2-1)p+p\\&=p+\text{compact operators},
 \end{align*}
since $[F_k,p]$ and $(F^2-1)$ are compact. Since $p=\Id_{pH_k}$ this shows that $pF_kp$ is Fredholm and so is $pF_k^+p$. Moreover if $p\simh q$ then $pF_k^+p\simh qF_k^+ q$ and the index is homotopy invariant by Theorem \eqref{thm: index_is_homotopy_invariant}. Hence $\Ind_F([p])$ only depends on $[p]$, as we wanted to show.
\end{proof}

\subsection{Differential Operators}
Let $(M,g)$ be a compact smooth Riemannian manifold of dimension $n$.

\begin{definition}
 A smooth vector bundle $E\to M$ is called \emph{Hermitian} if there exist an Hermitian structure (fiberwise inner product), that is a map $(-,-)\colon E\times E\to \Co^{\infty}(M)$ such that $$(-,-)_x\colon E_x\times E_x\to \Co^\infty(M)$$ is a inner product for all $x\in M$.
\end{definition}

\begin{definition}
 If $E\to M$ is an Hermitian bundle we denote with $\G^\infty(E)$ the space of smooth sections of $E$. We obtain an inner product on $\G^\infty(E)$ by setting, for $\varphi,\psi\in\G^\infty(E)$:$$\langle\varphi,\psi,\rangle=\int_M(\varphi(x),\psi(x))\mathrm{dvol}_g(x),$$ where $\mathrm{dvol}_g(x)=\sqrt{\det(g)}\mathrm{d}x^1\wedge\cdots\wedge\mathrm{d}x^n$ is the volume form of $M$. The completion of $\G^\infty(E)$ with respect to $\langle -,-\rangle$ is denoted by $L^2(E)$.
\end{definition}

\begin{definition}
 A \emph{1st order differential operator} $D$ on $E$ is a linear map $L^\infty(E)\to L^\infty(E)$ which in local coordinates has the form $$D=\sum_{j=1}^na_j\partial_{x_j}+b,\text{ where }a_j,b\in\G^\infty(\End(E)),$$
 where $\End(E)$ denotes the vector bundle with fibers $(\End(E))_x=\End (E_x)$.
\end{definition}

\begin{definition}
 The \emph{formal adjoint} $D^t$ of a 1st order differential operator $D$ is the unique 1st order differential operator on $E$ such that $$\langle D\varphi,\psi\rangle=\langle\varphi,D^t\psi\rangle, \text{ for all }\varphi,\psi\in L^2(E).$$
 
 \noindent $D$ is called \emph{formally self-adjoint} if $D=D^t$.
\end{definition}

\begin{remark}
 Since $L^2(E)$ is the completion of $L^\infty(E)$ we can think about a 1st order differential operator $D$ as a densely defined operator on $L^2(E)$ to obtain:
 \begin{enumerate}
  \item Since $D^t\subseteq D^\ast$,$D^\ast$ is densely defined and $D$ is closable.
  \item If $D$ is formally self-adjoint then $D$ is symmetric on $\dom (D)=L^\infty(E)$.
 \end{enumerate}
\end{remark}
In fact, since $M$ is compact and $D$ is first order we can say more:
\begin{proposition}
 If $D$ is as above and symmetric, then $D$ is essentially self-adjoint.
\end{proposition}
\begin{proof}[Proof (sketch):] The proof is based on the fact that if $\psi\in\dom (D)^\ast$ and $\supp(\psi)$ is compact, then $\psi\in\dom(\overline{D^t})$.
\end{proof}

\begin{definition}
 The \emph{principal symbol} of a 1st order differential operator $D$ is the map $\sigma_D\colon T^\ast M\to\End(E)$ given by $$\sigma_D(x,\xi)=[D,f](x),$$ where $x\in M$,$\xi\in T^\ast M$ and $f\in\Co^\infty(M)$ is such that $\mathrm{d}f=\xi$. \red{why is there such a $f$?} In local coordinates we have $$\xi=\sum\xi_j\mathrm{d}x^j\text{ and }\sigma_D(x,\xi)=\sum a_j\xi_j.$$ 
\end{definition}

\begin{definition}
 A 1st order differential operator $D$ is called \emph{elliptic} if for all $x\in M$ and all nonzero $\xi\in T^xM$, $\sigma_D(x,\xi)\colon E_x\to E_x$ is invertible.
\end{definition}

\noindent Let $W^{1,2}$ denote the first Sobolev space with inner product $\langle -,-\rangle$. \red{What is it?}. We have the following two facts that we assume without proof.
\begin{theorem}[Rellich's Lemma]The inclusion $W^{1,2}(E)\hookrightarrow L^2(E)$ is compact.\end{theorem}
\begin{lemma}[Gårding's inequality] There exist a constant $c>0$ such that $\langle \psi,\psi\rangle\leq c\langle\psi,\psi\rangle_D$ for $D$ elliptic.\end{lemma}

\noindent As a consequence we have $\dom(\overline{D})=W^{1,2}(E)$, with equivalent norms.
\begin{theorem}
 Let $D$ be a formally self-adjoint, elliptic 1st order differential operator. Then
 \begin{enumerate}
  \item $D$ is essentially self-adjoint.
  \item $\Co^\infty(M)\cdot\dom(D)\subseteq\dom(D)$ and $[D,f]$ is bounded for $f\in\Co^\infty(M)$.
  \item $(\overline{D}\pm i)^{-1}$ is compact.
 \end{enumerate}
\end{theorem}
\begin{proof}
 We have already seen $(1)$, and clearly $[D,f]=\sigma_D(\mathrm{d}f)$ is bounded.\\ To prove $(3)$ we look at $(\overline{D}\pm i)^{-1}\colon L^2(E)\to L^2(E)$ as the composition
 $$\begin{tikzcd}L^2(E)\arrow["(\overline{D}\pm i)^{-1}",r] & \dom(\overline{D}) \arrow["\Id",r] & W^{1,2}(E)\arrow["i",hook,r] & L^2(E)\end{tikzcd},$$ the first map is bounded since $D$ is self-adjoint \red{I don't see how that follows}, the second is bounded by Gårding's inequality, while the third is compact by Rellich's lemma hence the whole composition is compact.
\end{proof}

\begin{remark}
 We recall the spectral theorem for compact operators. Let $T$ be a compact normal operator on an Hilbert space $H$ (where normal means that $TT^\ast=T^\ast T$). Then $$H=H_0\oplus\bigoplus{n\in\N}H_{\mu_n},$$
 where $H_0=\ker T$ and $H_{\mu_n}$ ($\mu_n\neq 0$) is the finite dimensional eigenspace of $T$ with eigenvalue $\mu_n$. Moreover $|\mu_n|\to 0$ as $n\to\infty$ 
\end{remark}
\begin{corollary}
 With $D$ as before \red{maybe write it explicitely who's $D$} we have $L^2(E)=\overline{\bigoplus H_{\lambda_n}}$, where $H_{\lambda_n}$ is the finite dimensional eigenspace of $D$ with eigenvalue $\lambda_n$. Moreover $|\lambda_n|\to\infty$ as $n\to\infty$.
\end{corollary}
\begin{theorem}[Elliptic Regularity] If $\psi\in\dom(\overline{D})$ where $D$ is an elliptic operator and $\overline{D}\psi\in\G^\infty(E)$, then $\psi\in\G^\infty(E)=\dom(D)$. In particular $H_{\lambda_n}\subseteq\G^\infty(E)$ for each $\lambda_n$.
\end{theorem}
\subsection{Spectral Triples}
\begin{definition}
 A \emph{spectral triple} over a unital $C^\ast$-algebra $A$ is a triple $(\mathcal A,H,D)$ where $H$ is a Hilbert space equipped with a represention $\pi\colon A\to\B(H)$, a dense subalgebra $\mathcal A\subseteq A$ and a densely defined $D\colon\dom(D)\to H$ such that: 
 \begin{enumerate}
  \item $D$ is self-adjoint.
  \item $\mathcal A\cdot\dom(D)\subseteq\dom(D)$ and $[D,a]$ is bounded for $a\in\mathcal A$.
  \item $(D\pm i)^{-1}$ are compact operators.
 \end{enumerate}

\end{definition}












\section{Clifford Algebras and spin$^c$ Manifolds}
\subsection{Clifford Algebras}
\begin{definition}
 Given a finite dimensional real vector space $V$ with an inner product the Clifford algebra $\Cl(V)$ is defined to be the universal algebra generated by elements of $V$, with relations $vw+wv=-2\langle v,w\rangle$.
\end{definition}

\noindent Note that if $\{e_1,\ldots,e_n\}$ is an orthonormal basis for $V$, then $$\{e_{j_1}\cdots e_{j_k}\mid 1\leq i_1\leq\cdots\leq i_k,0\leq k\leq n\}$$ is a basis for $\Cl(V)$ (where the empty product is defined to be $1$, the multiplicative identity of the algebra). In particular $\dim\Cl(V)=\sum_{k=0}^n \binom{n}{k} = 2^n$.

\begin{definition}
 The \emph{complex clifford algebra} $\CL(V)$ is defined to be $\Cl(V)\otimes_\R\C$. It has an involution defined by $\lambda v_1\cdots v_k)^\ast=(-1)^k\overline{lambda} v_1\cdots v_k$, $\lambda\in\C$, $v_i\in V$.
\end{definition}

\noindent If $V=\R^n$ we write $\Cl_n$ and $CL_n$.
\begin{fact}[Classification of complex Clifford algebras] If $n=2m$ there is an isomorphism
$$\CL_n\to M_{2^m}(\C)=\End(\Delta_{2m}),$$ where $\Delta_{2m}$ is the "spinor module".

\noindent If $n=2m+1$ there is an isomorphism $$\CL_n\to M_{2^m}(\C)\oplus M_{2^m}(\C).$$
 \end{fact}



\section{Ideals of Compact Operators}
\subsection{The Schatten Norms and the Dixmier Ideal}
Let $H$ be an infinite dimensional separable Hilbert space. For $T\in\K(H)$ we consider the singular values $\{s_j(T)\}_{j=0}^\infty$ given by the decreasing sequence of eigenvalues of $|T|=(T^\ast T)^{1/2}$.

\begin{definition}
 For $p\in[1,\infty)$ we define the $p$-th Schatten class $\L^p$ as 
 $$\L^p(H)=\left\{T\in\K(H)\mid\sum_{j=0}^\infty s_j(T)^p<\infty\right\},$$
 with the norm $$\|T\|_p=\left(\sum_{j=0}^\infty s_j(T)^p\right)^{1/p}=(\Tr|T|^p)^{1/p}.$$
\end{definition}

\noindent In particular $\L^1(H)$ is the trace class, with $\|T\|=\Tr|T|$, while $\L^2(H)$ is the class of Hilbert-Schmidt operators. 

\noindent We have the following results:
\begin{fact}
 \begin{enumerate}
  \item $(\L^p(H),\|\cdot\|_p)$ is a Banach space.
  \item $\L^p(H)$ is a two sided ideal of $\K(H)$.
  \item Given an orthonormal basis $\{e_j\}\subseteq H$ we have that $\Tr T=\sum_{j=0}^\infty\langle e_j, Te_j\rangle$ is absolutely convergent if and only if $T\in\L^1(H)$.
  \item $\Tr\colon\L^1(H)\to\C$ is a bounded linear functional (independent of the choice of basis, by the previous fact) satisfying $\Tr(TS)=\Tr(ST)$ whenever both $TS$ and $ST$ are in $\L^1(H)$.
  \item For $p<r$ we have $\L^p(H)\subseteq\L^r(H)$.
  \item We have the following Hölder like inequality: for $1/p+1/q=1,T\in\L^p(H),S\in\L^q(H)$ then $TS\in\L^1(H)$ and $\Tr|TS|\leq\|T\|_p\|S\|_q$. This also holds for $p=1,q=\infty$ with $\L^\infty(H)=\B(H)$.
 \end{enumerate}
\end{fact}

\noindent In the following we write $\sigma_k(T)$ to denote the partial sums of the singular values of $T$, that is $\sigma_k(T)=\sum_{j=0}^{k-1}s_j(T)$, for all $k\geq 1$.

\begin{lemma}
 \begin{align*}
  \sigma_j(T)&=\inf\{\|T|_{E^\perp}\|\mid E\subseteq H, \dim E=j\}\\
  \sigma_k(T)&=\sup\{\|T|_{E}\|_1\mid E\subseteq H,\dim E=k\}.
 \end{align*}
\end{lemma}
\begin{corollary}\label{coro: Dixmier}
 $\sigma_k(S+T)\leq\sigma_k(S)+\sigma_k(T)$
\end{corollary}

\begin{definition}
 The \emph{Dixmier ideal} is defined as $$\L^{1+}(H)=\{T\in\K(H)\mid\|T\|_{1+}<+\infty\},$$ where $\|T\|_{1+}=\sup_{k\in\N}\frac{\sigma_k(T)}{\log k}$.
\end{definition}

\noindent By Corollary \eqref{coro: Dixmier} we have that $\|\cdot\|_{1+}$ is a norm and that $\L^{1+}(H)\subseteq\K(H)$ is a linear subspace. Furthermore we have the following inclusions (proving that the second inclusion is strict is hard):
$$\L^1(H)\subsetneq\L^{1+}(H)\subsetneq\bigcap_{\varepsilon>0}\L^{1+\varepsilon}(H).$$


\noindent We would now like to define the Dixmier trace as $$\Tr^+(T)=\lim_{k\to\infty}\frac{\sigma_k(T)}{\log k}$$ for $0<T\in\L^{1+}(H)$ but this limit does not exist in general! The solution will be to introduce a generalized limit $\omega$, strong enough to ensure that the Dixmier trace exists and to define $$\Tr_\omega(T)=\omega\left(\left(\frac{\sigma_k(T)}{\log k}\right)_{k\in\N}\right).$$ Let $\gamma_k=\sigma_k(T)/\log k$ and consider $0<S,T\in\L^{1+}(H)$. Thanks to Corollary \eqref{coro: Dixmier} we know that $\gamma_k(S+T)\leq\gamma_k(T)+\gamma_k(S)$ and we alos have
\begin{align*}
 \sigma_k(S)+\sigma_k(T)&=\sup_{\dim E_1=k}\Tr\left(S|_{E_1}\right)+\sup_{\dim E_2=k}\Tr\left(T|_{E_2}\right) \\
 &\leq\sup_{E_1,E_2}(\Tr(S|_{\langle E_1,E_2\rangle})+\Tr(T|_{\langle E_1,E_2\rangle}))\\
 &\leq\sup_{\dim E=2k}(\Tr((S+T)|_E))\\
 &=\sigma_{2k}(S+T)
\end{align*}
thus $\gamma_k(T)+\gamma_k(S)\leq\frac{\log 2k}{\log k}\gamma_{2k}(S+T)$ so if we want $\Tr_\omega$ to be additive we need to ensure that $\omega((\gamma_{2k})_{k\in\N})=\omega((\gamma_k)_{k\in\N})$.

\begin{definition}
 A \emph{generalized limit} is a positive linear functional $\omega\colon\ell^\infty\to\Bbb C$ satisfying:
 \begin{enumerate}
  \item $\omega(1,1,1,\ldots)=1$.
  \item If $\lim_{k\to\infty} a_k=0$, then $\omega((a_k)_{k\in\N})=0$.
  \item $\omega(a_1,a_2,a_3,\ldots)=\omega(a_1,a_1,a_2,a_2,a_3,a_3,\ldots)$.
 \end{enumerate}
\end{definition}
\red{Does taking $\omega$ to be the limit along a nonprincipal ultrafilter on $\Bbb N$ work?}

\begin{lemma}
 Generalized limits exists and they also satisfy 
\begin{enumerate}\setcounter{enumi}{3}
 \item $\liminf_{k\to\infty}a_k\leq\omega((a_k)_{k\in\N})\leq\limsup_{k\to\infty}a_k$
\end{enumerate}
\end{lemma}

\begin{proposition}
 For any generalized limit $\omega\colon\ell^\infty\to\C$ setting $\Tr_\omega(T)=\omega((\gamma_k(T))_{k\in\N})$ for $0\leq T\in\L^{1+}(H)$ defined a trace on $\L^{1+}(H)$, that is a positive linear functional $\Tr_\omega\colon\L^{1+}(H)\to\C$ satisfying $\Tr(TT^\ast)=\Tr(T^\ast T)$. Moreover $\Tr_\omega$ vanishes on $\L^1(H)$.
\end{proposition}
\begin{proof}
 We begin by proving that if $T\in\L^1(H)$, $0<T$, then $\Tr_\omega(T)=0$. This is clear, because $T\in\L^1(H)$ implies that $\sigma_k(T)$ is bounded, so $\gamma_k(T)\to 0$ and by property (2) of generalized limits we have $\omega(\gamma_k))=0$.
 
 \noindent Now if $T\in\L^{1+}(H)$, writing $\gamma_k$ for $\gamma_k(T)$, we have 
 \begin{align*}
  \omega(\gamma_{2k})&=\omega(\gamma_2,\gamma_2,\gamma_4,\gamma_4,\ldots)\\
  &=\omega(\gamma_k)+\omega(\gamma_2-\gamma_1,0,\gamma_4-\gamma_3,\ldots)\\
  &=\omega(\gamma_k)
 \end{align*}
because $\gamma_{2k}-\gamma_{2k-1}=\left(1-\frac{\log 2k}{\log 2k-1}\right)\gamma_{2k}+\frac1{\log 2k-1}s_{2k-1}(T)$.
Thus $\Tr_\omega$ is linear on positive elements. We briefly sketch how to conclude:

\noindent First of all $\Tr_\omega$ can be extended by linearity to the whole of $\L^{1+}(H)$. Then the fact that $\Tr_\omega(TT^\ast)=\Tr(T^\ast T)$ follows from the fact that $s_j(TT^\ast)=s_j(T^\ast T)$ whenever both are nonzero.
\end{proof}

\begin{definition}
 An operator $T\in\L^{1+}(H)$ such that $\lim_{k\to\infty}\gamma_k(T)$ exists is called \emph{measurable} and we write 
 $$\Tr^+(T)=\Tr_\omega(T)=\lim_{k\to\infty}\gamma_k(T).$$
 Note that in this case $\Tr_\omega(T)$ does not depend on the specific $\omega$ we picked.
\end{definition}

\begin{proposition}
 We have an Hölder inequality for the Dixmier trace as well. Let $1/p+1/q=1$, $|T|^p,|S|^q\in\L^{1+}(H)$. Then $TS\in\L^{1+}(H)$ and $$\Tr_\omega|TS|\leq(\Tr_\omega|T|^p)^{1/p}(\Tr_\omega|S|^q)^{1/q}.$$
\end{proposition}

\begin{example}
 For example consider $M=T^n$, the $n$-dimensional torus and consider the Laplacian $\Delta_n=-\sum_{j=1}^n\partial^2_j$, whose eigenvalues (all with multiplicity $1$) are $\Z^n$. For $R>>0$ one can estimate 
 $$\sum_{\substack{z\in\Z^n\\ \|z\|<R}}(1+\|z\|)^{-s}\sim\int_0^R(1+r^2)^{-s}\vol(S^{n-1})r^{n-1}\mathrm{d}r\sim\vol(S^{n-1})\int_0^Rr^{n-2s-1}\mathrm{d}r$$
 
 \noindent Now let $N_r=\#\{z\in\Z^n\mid \|z\|\leq R\}$. Since $\log N_r\sim n\log R$, we have, for $s>n/2$
 $$\gamma_{N_r}((1+\Delta_n)^{-s})=\frac1{\log N_r}\sum_{\|z\|\leq R}(1+\|z\|^2)^{-s}\to 0,\text{ as }R\to+\infty$$
 
 \noindent For $s=n/2$ we get instead $\gamma_{N_r}((1+\Delta_n)^{-s})\sim\frac1n\vol(S^{n-1})$. Thus $(1+\Delta_n)^{-n/2}\in\L^{1+}$ and $$\Tr^+((1+\Delta_n)^{-n/2})=\frac1n\vol(S^{n-1}).$$
\end{example}

\subsection{Summability}
\begin{definition}
 A spectral triple $\sptr$ is called \emph{finitely summable} if there is $s_0>0$ such that $(1+D^2)^{-s_0/2}\in\L^1(H)$. Note that in this case we also have $(1+D^2)^{-s/2}\in\L^1(H)$ for all $s>s_0$ and we can define the \emph{spectral dimension} as 
 $$\rho=\inf\{s\in[0,\infty)\mid (1+D^2)^{-s/2}\in\L^1(H)\}.$$
\end{definition}

\begin{definition}
 A spectral triple $\sptr$ is called \emph{$p^+$-summable} if $(1+D^2)^{-p/2}\in\L^{1+}(H)$.
\end{definition}

\noindent Note that if $\sptr$ is $p^+$-summable then it is also finitely summable with spectral dimension $p$.

\begin{example}
 If $D$ is a Dirac operator on a Clifford module $E$ over an $n$-dimensional $\spinc$ manifold then $(1+D^2)^{-n/2}$ is a classical pseudodifferential operator of order $-n$.
\end{example}

\noindent In the next lecture we will see Connes trace theorem: $(1+D^2)^{-n/2}\in\L^{1+}(H)$, hence $(\Co^\infty(M),L^2(E),D)$ is $n^+$-summable.











\section{Connes Trace Theorem}
\subsection{The Wodzicki Residue}
Let $\psi^r(M)$ denote the space of pseudodifferential operators of order $r$ on a Riemannian manifold $M$. We also write $$\psi^\infty(M)=\bigcup_{r\in\R}\psi^r(M)\quad\quad\psi^{-\infty}(M)=\bigcap_{r\in\R}\psi^r(M),$$
operators in $\psi^{-\infty}(M)$ are known as \emph{smoothing operators}.

\noindent Note that, since $M$ is compact, $\psi^\infty(M)$ is an algebra. We say that a pseudodifferential operator is classical if its symbol $\sigma(P)\in\G^\infty(T^\ast M,\End(E))$ \red{I'm not sure what this notation means, shouldn't we have $\sigma(P)\colon T^\ast M\to\End(E)$?} has asymptotic expansion $$\sigma(P)(x,\xi)\sim\sum_{j=0}^\infty\sigma_{n-j}(P)(x,\xi),$$
where each $\sigma_{n-j}$ is homogeneous in $\xi$ and of order $n-j$.

\noindent We have that the symbol determines an operator up to a smoothing operator, meaning that there is a map $\Op\colon\G^\infty(T^\ast M,\End(E))\to\psi^\infty(M,E)$ such that $\Op(\sigma(P))-P\in\psi^{-\infty}(M)$.

\begin{theorem}[Wodzicki]
 Let $n=\dim M$. Given a classical pseudodifferential operator $P$ there exist a \emph{Wodzicki residue density} given locally by $$\wres_x(P)=\int_{S^{n-1}}\Tr(\sigma_{-n}(P)(x,\xi))\mathrm{d}\mu(\xi)\mathrm{d}^nx,$$
 where $$\mathrm{d}\mu(\xi)=\sum_{j=1}^n(-1)^j\xi_j\mathrm{d}\xi_1\wedge\cdots\wedge\widehat{\mathrm{d}\xi_j}\wedge\cdots\wedge\mathrm{d}\xi_n,$$
 where this notation means that we skip $\mathrm{d}\xi_j$. Using this we obtain the \emph{Wodzicki residue} $$\Wres(P)=\int_M\wres_x(P).$$
 The Wodzicki residue is a trace on the classical pseudodifferential operators. If $n>1$ it is the only trace up to rescaling.
\end{theorem}
\begin{proof}
 We won't prove the theorem but note that the main step is proving that $\wres_x(P)$ is independent of the choice of coordinates.
\end{proof}

\begin{proposition}
 Let $D$ be a Dirac operator on a Clifford module $E$ of rank $k$ on a compact Riemannian $\spinc$-manifold of dimension $n$. Then $$\wres_x((1+D^2)^{-n/2})=k\vol(S^{n-1})\mathrm{dvol}_g(x).$$
\end{proposition}

\begin{proof}[Proof (sketch):] We have $$\sigma_n(1+D^2)^{-n/2}=(\sigma_2(1+D^2))^{-n/2}=(c(\xi)^2)^{-n/2}=g(\xi,\xi)^{-n/2}\cdot\Id_k.$$
 We can apply a change of coordinates $x\leadsto y$,$\xi\leadsto\eta$ such that $\sigma_{-n}=|\eta|^{-n}$ and $\mathrm{d}^ny=\sqrt{\det g}\mathrm{d}^nx$. 
 
 \noindent Then $$\wres_x(1+D^2)^{-n/2}=\Tr(\Id_k)\int_{S^{n-1}}|\eta|^{-n}\mathrm{d}^n\eta\mathrm{d}^ny=k\vol(S^{n-1})\sqrt{\det g}\mathrm{d}^nx.$$
\end{proof}

\begin{example}
 Consider $M=\T^n$ (the $n$-torus) with the trivial $\spinc$ structure: $E=M\times\C^{2^m}$, with $n=2m$ or $n=2m+1$. Then $D_n^2=\Delta_n\Id_{2^n}$ and we get $$\Wres(1+\Delta_n)^{-n/2}=\frac1{2^n}\Wres(1+D^2)^{-n/2}=\vol(S^{n-1})(2\pi)^n.$$
\end{example}


\begin{theorem}[Connes Trace theorem]
 Let $M$ be a compact $n$-dimensional Riemannian manifold, let $E\to M$ be a vector bundle and let $P\in\psi^{-n}(M,E)$ be a classical pseudodifferential operator. Then $P\in\L^{1+}(L^2(E))$, $P$ is measurable and $\Tr^+(P)=\frac{1}{n(2\pi)^n}\Wres(P)$.
 
 \noindent Hence $\Tr^+((1+\Delta_n)^{-n/2})=\frac1{n(2\pi)^n}\Wres(1+\Delta_n)^{-n/2}$.
\end{theorem}

\begin{proof}[Proof (Sketch):]
 We won't give a detailed proof, but only an outline of the steps to follow to prove the theorem.
 \begin{enumerate}
  \item By a partition of unity argument it suffices to prove the theorem for $M=\T^n$.
  \item $P=\underbrace{P(1+\Delta_n)^{n/2}}_{\text{zeroth order bounded}}\underbrace{(1+\Delta_n)^{-n/2}}_{\in\L^{1+}(L^2(E))}\in\L^{1+}(L^2(E))$
  \item Since $\Tr^+(1+\Delta_n)^{-s/2}=0$ for all $s>n$ we have $\Tr(Q)=0$ for every $Q$ of order $-s<-n$. In particular this holds for $Q=\Op(\sigma(P)-\sigma_{-n}(P))$. Hence $\Tr^+(P)$ depends only on $\sigma_{-n}(P)$.
  \item The map $\sigma_{-n}(P)\to\Tr^+(P)$ determines a measure on $\Co^\infty(S^\ast M)$, where $S^\ast M=\{\xi\in T^\ast M\mid |\xi|=1\}$ is the so called cosphere bundle. Since $\Tr^+$ is invariant under unitary transformations this measure must be invariant under rotations, hence it must be proportional to the volume form.
  \item The constant of proportionality can be worked by computed an example, like we did above for $P=(1+\Delta_n)^{-n/2}$.
 \end{enumerate}
\end{proof}

\subsection{Hochschild (Co)Homology}
Given an algebra $A$ (in particular we will look at $A=\Co^\infty(M)$) we can form a chain complex of algebras $(\Co_\bullet(A),b)$ where $\Co_k(A)=A^{\otimes(k+1)}$ and the boundary map $b\colon \Co_{k+1}(A)\to\Co_k(A)$ is given by 
$$b(a_0\otimes\cdots\otimes a_k)=\sum_{j=0}^{k-1}(-1)^ja_0\otimes\cdots\otimes a_ja_{j+1}\otimes\cdots\otimes a_k+(-1)^k a_ka_0\otimes\cdots\otimes a_{k-1},$$
and $b=0$ on $\Co_0(A)$. We actually have $b^2=0$, but checking this is a messy computation that we won't carry out explicitely.

\begin{definition}
 The \emph{Hochschild Homology} $HH_\bullet(A)$ is the homology of the chain complex $(\Co_\bullet(A),b)$, explicitely $HH_k(A)=Z_k(A)/B_k(A)$ where 
 \begin{align*}
  Z_k(A)&=\ker(b\colon\Co_k(A)\to\Co_{k-1}(A))\\
  B_k(A)&=\mathrm{Im}(b\colon\Co_{k+1}(A)\to\Co_k(A))
 \end{align*}
\end{definition}

\begin{example}
 If $A\simeq\C$ then $\Co_n(\C)\simeq\C$ and $$b=\sum_{j=0}^k(-1)^j=\begin{cases} 1 & k \text{ is even} \\ 0 & k\text{ is odd}\end{cases}.$$
 
 \noindent So we obtain \begin{tikzcd} \C \arrow[r,"0"] & \C \arrow[r,"1"] & \C \arrow[r,"0"] & \C\arrow[r] & 0\end{tikzcd} and we get $HH_0(\C)=\C$ and $HH_k(\C)=0$ for $k>0$.
\end{example}

\begin{lemma}
 For every algebra $A$ we have $HH_0(A)=A/[A,A]$, so in particular $HH_0(A)\simeq A$ if $A$ is commutative.
\end{lemma}

\begin{example}
 If $A=\Co^\infty(M)$ then $\Co_k(A)\simeq\Co^\infty(M)^k$ and 
 $$(bF)(x_0,\ldots,x_{k-1})=\sum_{j=0}^{k-1}(-1)^j F(x_0,\ldots,x_j,x_{j+1},\ldots,x_{k-1})+(-1)^kF(x_0,\ldots,x_{k-1},x_0).$$
 
 \noindent There is a natural map $\Omega^k(M)\to HH_k(\Co^\infty(M))$ given locally by 
 $$f\mathrm{d}x^1\wedge\cdots\wedge\mathrm{d}x^k\mapsto\frac1{k!}\sum_{\pi\in S_k}(-1)^{\mathrm{sign}(\pi)}f\otimes x^{\pi(1)}\otimes\cdots\otimes x^{\pi(k)}.$$
 
 \noindent We have in particular $HH_0(\Co^\infty(M))\simeq\Co^\infty(M)$.
\end{example}

\noindent Connes proved that in fact $HH_\bullet(\Co^\infty(M))\simeq\Omega^\bullet(M)$, so that de Rham cohomology can be recovered from Hochschild homology. More precisely Connes proved a dual version we will present shortly, after defining Hochschild cohomology.

\begin{definition}
 Let $\Co^k(A)$ denote the \emph{Hochschild $k$-cochains}, that is linear forms on $\Co_k(A)$, or linear functionals on $A^{\otimes(k+1)}$. The coboundary operator $b\colon\Co^k(A)\to\Co^{k+1}(A)$ is given by 
 $$b\varphi(a_0,\ldots,a_{k+1})=\sum_{j=0}^k(-1)^j\varphi(a_0,\ldots,a_ja_{j+1},\ldots a_{k+1})+(-1)^k\varphi(a_{k-1},a_0,\ldots,a_k).$$
 
 \noindent As usual Hochschild cohomology is defined as $HH^\bullet(A)=H^\bullet(\Co^k(A))$.
\end{definition}

\noindent Note that an Hochschild $0$-cocycle $\tau$ is a trace on $A$, since $\tau(ca-ac)=b\tau(a,c)=0$.

\begin{definition}
 For a compact manifold $M$ let $D_k(M)=\{\text{continuous linear functionals }\Omega^k(M)\to\C\}$ be the space of $k$-De Rham currents on $M$. For $C\in D_k(M)$ and $\alpha\in\Omega^k(M)$ we have a pairing given by $\int_C\alpha\in\C$. The boundary of $C$ is the $k-1$ current $\partial C$ given by $$\int_{\partial C}\beta=\int_C\mathrm{d}\beta,\quad\beta\in\Omega^{k-1}(M).$$
\end{definition}

\begin{theorem}[Connes]
 $HH^\bullet(\Co^\infty(M))\simeq D_\bullet(M)$.
\end{theorem}

\subsection{Regularity and Connes Characters Formula}
Given a spectral triple $\sptr$ we write $\delta(T)=[|D|,T]$, for $T\in\dom\delta$, where 
$$\dom\delta=\{T\in\B(H)\mid T\dom D\to\dom D\text{ and } [|D|,T]\text{ is bounded on }\dom D\}.$$

\begin{definition}
 A spectral triple $\sptr$ is called \emph{regular} if for each $a\in A$ and each $k\in\N$, both $a$ and $[D,a]$ belong to $\dom\delta^k$.
\end{definition}

\begin{definition}
 If $A$ is a $C^\ast$-algebra then a $\ast$-subalgebra $\A\subseteq A$ is called a \emph{pre-$C^\ast$-algebra} if
 \begin{enumerate}
  \item $\A$ is complete with respect to some locally convex topology finer than the norm topology of $A$.
  \item If $a\in\A$ is invertible in $A$, then $a^{-1}\in\A$.
 \end{enumerate}
\end{definition}

\begin{example}
 The standard example, which we have encountered multipled times already, is $\Co^\infty(M)\subseteq\Co(M)$.
\end{example}

\noindent Given a spectral triple $\sptr$ we write $\G$ for the $\Z_2$-grading if $\sptr$ is even, and $\G=1$ if it is odd.

\begin{lemma}
 If $\sptr$ is $n^+$-summable and $\omega$ is a generalized limit, then 
 \begin{enumerate}
  \item For all $a\in\A$ and $T\in\B(H)$ we have $\Tr_\omega([T,a]|D|^{-n})=0$.
  \item $\varphi^\omega_D(a_0,\ldots,a_n)=\Tr_\omega(\G a_0[D,a_1]\ldots[D,a_n]|D|^{-n})$ defines an Hochschild $n$-cocycle
 \end{enumerate}
 
 \noindent Note that the above definition doesn't really work if $D$ is not invertible, if $D$ isn't invertible we should either remove the finite dimensional $\ker D$ from $H$, or use $1+D^2$ instead.
\end{lemma}

\begin{proof}[Proof (Sketch):]
 By Hölder's inequality we have $$|\Tr_\omega([T,a]|D|^{-n})=|\Tr_\omega(T[|D|^{-n},a])\leq\|T\|\cdot\Tr|[|D|^{-n},a]|=0$$
 (the last equality is very tedious to check). Furthermore we have
 $$b\varphi^\omega_D(a_0,\ldots,a_{n+1})=(-1)^n\Tr_\omega([\G a_0[D,a_1]\ldots[D.a_n],a_{n+1}]|D|^{-n})=0,\quad\text{ by (1)}.$$
\end{proof}

\begin{lemma}
 Let $F$ denote $D|D|^{-1}$. If $\sptr$ is $p$-summable then $[F,a]\in\L^p(H)$ for all $a\in A$.
\end{lemma}

\begin{corollary}
 If $\sptr$ is $n^+$ summable then $[F,a]\in\L^{n+1}(H)$ and $[F,a_0]\ldots[F,a_n]\in\L^{1}(H)$.\red{who are the $a_i$ here?}
\end{corollary}

\begin{theorem}[Connes character formula]
Let $\sptr$ be a recular and $n^+$ summable spectral triple, $\omega$ a generalised limit. If $c=\sum_j a_{0_j}\otimes\ldots\otimes a_{n_j}$ \red{the indices confuse me here} is a Hochschild $n$-cycle, then 
$$\langle\varphi^\omega_D,c\rangle=\sum_j\Tr(\G F[F,a_{0_j}]\ldots[F,a_{n_j}]),$$
in particular $\langle\varphi^\omega_D,c\rangle$ is independent of $\omega$.
 
\end{theorem}














\end{document}
