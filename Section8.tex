\section{Connes Trace Theorem}
\subsection{The Wodzicki Residue}
Let $\psi^r(M)$ denote the space of pseudodifferential operators of order $r$ on a Riemannian manifold $M$. We also write $$\psi^\infty(M)=\bigcup_{r\in\R}\psi^r(M)\quad\quad\psi^{-\infty}(M)=\bigcap_{r\in\R}\psi^r(M),$$
operators in $\psi^{-\infty}(M)$ are known as \emph{smoothing operators}.

\noindent Note that, since $M$ is compact, $\psi^\infty(M)$ is an algebra. We say that a pseudodifferential operator is classical if its symbol $\sigma(P)\in\G^\infty(T^\ast M,\End(E))$ \red{I'm not sure what this notation means, shouldn't we have $\sigma(P)\colon T^\ast M\to\End(E)$?} has asymptotic expansion $$\sigma(P)(x,\xi)\sim\sum_{j=0}^\infty\sigma_{n-j}(P)(x,\xi),$$
where each $\sigma_{n-j}$ is homogeneous in $\xi$ and of order $n-j$.

\noindent We have that the symbol determines an operator up to a smoothing operator, meaning that there is a map $\Op\colon\G^\infty(T^\ast M,\End(E))\to\psi^\infty(M,E)$ such that $\Op(\sigma(P))-P\in\psi^{-\infty}(M)$.

\begin{theorem}[Wodzicki]
 Let $n=\dim M$. Given a classical pseudodifferential operator $P$ there exist a \emph{Wodzicki residue density} given locally by $$\wres_x(P)=\int_{S^{n-1}}\Tr(\sigma_{-n}(P)(x,\xi))\mathrm{d}\mu(\xi)\mathrm{d}^nx,$$
 where $$\mathrm{d}\mu(\xi)=\sum_{j=1}^n(-1)^j\xi_j\mathrm{d}\xi_1\wedge\cdots\wedge\widehat{\mathrm{d}\xi_j}\wedge\cdots\wedge\mathrm{d}\xi_n,$$
 where this notation means that we skip $\mathrm{d}\xi_j$. Using this we obtain the \emph{Wodzicki residue} $$\Wres(P)=\int_M\wres_x(P).$$
 The Wodzicki residue is a trace on the classical pseudodifferential operators. If $n>1$ it is the only trace up to rescaling.
\end{theorem}
\begin{proof}
 We won't prove the theorem but note that the main step is proving that $\wres_x(P)$ is independent of the choice of coordinates.
\end{proof}

\begin{proposition}
 Let $D$ be a Dirac operator on a Clifford module $E$ of rank $k$ on a compact Riemannian $\spinc$-manifold of dimension $n$. Then $$\wres_x((1+D^2)^{-n/2})=k\vol(S^{n-1})\mathrm{dvol}_g(x).$$
\end{proposition}

\begin{proof}[Proof (sketch):] We have $$\sigma_n(1+D^2)^{-n/2}=(\sigma_2(1+D^2))^{-n/2}=(c(\xi)^2)^{-n/2}=g(\xi,\xi)^{-n/2}\cdot\Id_k.$$
 We can apply a change of coordinates $x\leadsto y$,$\xi\leadsto\eta$ such that $\sigma_{-n}=|\eta|^{-n}$ and $\mathrm{d}^ny=\sqrt{\det g}\mathrm{d}^nx$. 
 
 \noindent Then $$\wres_x(1+D^2)^{-n/2}=\Tr(\Id_k)\int_{S^{n-1}}|\eta|^{-n}\mathrm{d}^n\eta\mathrm{d}^ny=k\vol(S^{n-1})\sqrt{\det g}\mathrm{d}^nx.$$
\end{proof}

\begin{example}
 Consider $M=\T^n$ (the $n$-torus) with the trivial $\spinc$ structure: $E=M\times\C^{2^m}$, with $n=2m$ or $n=2m+1$. Then $D_n^2=\Delta_n\Id_{2^n}$ and we get $$\Wres(1+\Delta_n)^{-n/2}=\frac1{2^n}\Wres(1+D^2)^{-n/2}=\vol(S^{n-1})(2\pi)^n.$$
\end{example}


\begin{theorem}[Connes Trace theorem]
 Let $M$ be a compact $n$-dimensional Riemannian manifold, let $E\to M$ be a vector bundle and let $P\in\psi^{-n}(M,E)$ be a classical pseudodifferential operator. Then $P\in\L^{1+}(L^2(E))$, $P$ is measurable and $\Tr^+(P)=\frac{1}{n(2\pi)^n}\Wres(P)$.
 
 \noindent Hence $\Tr^+((1+\Delta_n)^{-n/2})=\frac1{n(2\pi)^n}\Wres(1+\Delta_n)^{-n/2}$.
\end{theorem}

\begin{proof}[Proof (Sketch):]
 We won't give a detailed proof, but only an outline of the steps to follow to prove the theorem.
 \begin{enumerate}
  \item By a partition of unity argument it suffices to prove the theorem for $M=\T^n$.
  \item $P=\underbrace{P(1+\Delta_n)^{n/2}}_{\text{zeroth order bounded}}\underbrace{(1+\Delta_n)^{-n/2}}_{\in\L^{1+}(L^2(E))}\in\L^{1+}(L^2(E))$
  \item Since $\Tr^+(1+\Delta_n)^{-s/2}=0$ for all $s>n$ we have $\Tr(Q)=0$ for every $Q$ of order $-s<-n$. In particular this holds for $Q=\Op(\sigma(P)-\sigma_{-n}(P))$. Hence $\Tr^+(P)$ depends only on $\sigma_{-n}(P)$.
  \item The map $\sigma_{-n}(P)\to\Tr^+(P)$ determines a measure on $\Co^\infty(S^\ast M)$, where $S^\ast M=\{\xi\in T^\ast M\mid |\xi|=1\}$ is the so called cosphere bundle. Since $\Tr^+$ is invariant under unitary transformations this measure must be invariant under rotations, hence it must be proportional to the volume form.
  \item The constant of proportionality can be worked by computed an example, like we did above for $P=(1+\Delta_n)^{-n/2}$.
 \end{enumerate}
\end{proof}

\subsection{Hochschild (Co)Homology}
Given an algebra $A$ (in particular we will look at $A=\Co^\infty(M)$) we can form a chain complex of algebras $(\Co_\bullet(A),b)$ where $\Co_k(A)=A^{\otimes(k+1)}$ and the boundary map $b\colon \Co_{k+1}(A)\to\Co_k(A)$ is given by 
$$b(a_0\otimes\cdots\otimes a_k)=\sum_{j=0}^{k-1}(-1)^ja_0\otimes\cdots\otimes a_ja_{j+1}\otimes\cdots\otimes a_k+(-1)^k a_ka_0\otimes\cdots\otimes a_{k-1},$$
and $b=0$ on $\Co_0(A)$. We actually have $b^2=0$, but checking this is a messy computation that we won't carry out explicitely.

\begin{definition}
 The \emph{Hochschild Homology} $HH_\bullet(A)$ is the homology of the chain complex $(\Co_\bullet(A),b)$, explicitely $HH_k(A)=Z_k(A)/B_k(A)$ where 
 \begin{align*}
  Z_k(A)&=\ker(b\colon\Co_k(A)\to\Co_{k-1}(A))\\
  B_k(A)&=\mathrm{Im}(b\colon\Co_{k+1}(A)\to\Co_k(A))
 \end{align*}
\end{definition}

\begin{example}
 If $A\simeq\C$ then $\Co_n(\C)\simeq\C$ and $$b=\sum_{j=0}^k(-1)^j=\begin{cases} 1 & k \text{ is even} \\ 0 & k\text{ is odd}\end{cases}.$$
 
 \noindent So we obtain \begin{tikzcd} \C \arrow[r,"0"] & \C \arrow[r,"1"] & \C \arrow[r,"0"] & \C\arrow[r] & 0\end{tikzcd} and we get $HH_0(\C)=\C$ and $HH_k(\C)=0$ for $k>0$.
\end{example}

\begin{lemma}
 For every algebra $A$ we have $HH_0(A)=A/[A,A]$, so in particular $HH_0(A)\simeq A$ if $A$ is commutative.
\end{lemma}

\begin{example}
 If $A=\Co^\infty(M)$ then $\Co_k(A)\simeq\Co^\infty(M)^k$ and 
 $$(bF)(x_0,\ldots,x_{k-1})=\sum_{j=0}^{k-1}(-1)^j F(x_0,\ldots,x_j,x_{j+1},\ldots,x_{k-1})+(-1)^kF(x_0,\ldots,x_{k-1},x_0).$$
 
 \noindent There is a natural map $\Omega^k(M)\to HH_k(\Co^\infty(M))$ given locally by 
 $$f\mathrm{d}x^1\wedge\cdots\wedge\mathrm{d}x^k\mapsto\frac1{k!}\sum_{\pi\in S_k}(-1)^{\mathrm{sign}(\pi)}f\otimes x^{\pi(1)}\otimes\cdots\otimes x^{\pi(k)}.$$
 
 \noindent We have in particular $HH_0(\Co^\infty(M))\simeq\Co^\infty(M)$.
\end{example}

\noindent Connes proved that in fact $HH_\bullet(\Co^\infty(M))\simeq\Omega^\bullet(M)$, so that de Rham cohomology can be recovered from Hochschild homology. More precisely Connes proved a dual version we will present shortly, after defining Hochschild cohomology.

\begin{definition}
 Let $\Co^k(A)$ denote the \emph{Hochschild $k$-cochains}, that is linear forms on $\Co_k(A)$, or linear functionals on $A^{\otimes(k+1)}$. The coboundary operator $b\colon\Co^k(A)\to\Co^{k+1}(A)$ is given by 
 $$b\varphi(a_0,\ldots,a_{k+1})=\sum_{j=0}^k(-1)^j\varphi(a_0,\ldots,a_ja_{j+1},\ldots a_{k+1})+(-1)^k\varphi(a_{k-1},a_0,\ldots,a_k).$$
 
 \noindent As usual Hochschild cohomology is defined as $HH^\bullet(A)=H^\bullet(\Co^k(A))$.
\end{definition}

\noindent Note that an Hochschild $0$-cocycle $\tau$ is a trace on $A$, since $\tau(ca-ac)=b\tau(a,c)=0$.

\begin{definition}
 For a compact manifold $M$ let $D_k(M)=\{\text{continuous linear functionals }\Omega^k(M)\to\C\}$ be the space of $k$-De Rham currents on $M$. For $C\in D_k(M)$ and $\alpha\in\Omega^k(M)$ we have a pairing given by $\int_C\alpha\in\C$. The boundary of $C$ is the $k-1$ current $\partial C$ given by $$\int_{\partial C}\beta=\int_C\mathrm{d}\beta,\quad\beta\in\Omega^{k-1}(M).$$
\end{definition}

\begin{theorem}[Connes]
 $HH^\bullet(\Co^\infty(M))\simeq D_\bullet(M)$.
\end{theorem}

\subsection{Regularity and Connes Characters Formula}
Given a spectral triple $\sptr$ we write $\delta(T)=[|D|,T]$, for $T\in\dom\delta$, where 
$$\dom\delta=\{T\in\B(H)\mid T\dom D\to\dom D\text{ and } [|D|,T]\text{ is bounded on }\dom D\}.$$

\begin{definition}
 A spectral triple $\sptr$ is called \emph{regular} if for each $a\in A$ and each $k\in\N$, both $a$ and $[D,a]$ belong to $\dom\delta^k$.
\end{definition}

\begin{definition}
 If $A$ is a $C^\ast$-algebra then a $\ast$-subalgebra $\A\subseteq A$ is called a \emph{pre-$C^\ast$-algebra} if
 \begin{enumerate}
  \item $\A$ is complete with respect to some locally convex topology finer than the norm topology of $A$.
  \item If $a\in\A$ is invertible in $A$, then $a^{-1}\in\A$.
 \end{enumerate}
\end{definition}

\begin{example}
 The standard example, which we have encountered multipled times already, is $\Co^\infty(M)\subseteq\Co(M)$.
\end{example}

\noindent Given a spectral triple $\sptr$ we write $\G$ for the $\Z_2$-grading if $\sptr$ is even, and $\G=1$ if it is odd.

\begin{lemma}
 If $\sptr$ is $n^+$-summable and $\omega$ is a generalized limit, then 
 \begin{enumerate}
  \item For all $a\in\A$ and $T\in\B(H)$ we have $\Tr_\omega([T,a]|D|^{-n})=0$.
  \item $\varphi^\omega_D(a_0,\ldots,a_n)=\Tr_omega(\G a_0[D,a_1]\ldots[D,a_n]|D|^{-n})$ defines an Hochschild $n$-cocycle
 \end{enumerate}
 
 \noindent Note that the above definition doesn't really work if $D$ is not invertible, if $D$ isn't invertible we should either remove the finite dimensional $\ker D$ from $H$, or use $1+D^2$ instead.
\end{lemma}

\begin{proof}[Proof (Sketch):]
 By Hölder's inequality we have $$|\Tr_\omega([T,a]|D|^{-n})=|\Tr_\omega(T[|D|^{-n},a])\leq\|T\|\cdot\Tr|[|D|^{-n},a]|=0$$
 (the last equality is very tedious to check). Furthermore we have
 $$b\varphi^\omega_D(a_0,\ldots,a_{n+1})=(-1)^n\Tr_\omega([\G a_0[D,a_1]\ldots[D.a_n],a_{n+1}]|D|^{-n})=0,\quad\text{ by (1)}.$$
\end{proof}

\begin{lemma}
 Let $F$ denote $D|D|^{-1}$. If $\sptr$ is $p$-summable then $[F,a]\in\L^p(H)$ for all $a\in A$.
\end{lemma}

\begin{corollary}
 If $\sptr$ is $n^+$ summable then $[F,a]\in\L^{n+1}(H)$ and $[F,a_0]\ldots[F,a_n]\in\L^{1}(H)$.\red{who are the $a_i$ here?}
\end{corollary}

\begin{theorem}[Connes character formula]
Let $\sptr$ be a recular and $n^+$ summable spectral triple, $\omega$ a generalised limit. If $c=\sum_j a_{0_j}\otimes\ldots\otimes a_{n_j}$ \red{the indices confuse me here} is a Hochschild $n$-cycle, then 
$$\langle\varphi^\omega_D,c\rangle=\sum_j\Tr(\G F[F,a_{0_j}]\ldots[F,a_{n_j}]),$$
in particular $\langle\varphi^\omega_D,c\rangle$ is independent of $\omega$.
 
\end{theorem}













